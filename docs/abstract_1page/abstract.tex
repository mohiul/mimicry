\documentclass[letterpaper]{article}
\usepackage{natbib,alifeconf}
%Graphics package
\usepackage{graphicx}
\DeclareGraphicsExtensions{.png,.jpg,.pdf}
%For multirow feature to draw tables
\usepackage{multirow}
%Middle allignment in tables.
\usepackage{array}
%For Mathematics
\usepackage{amsmath}
\numberwithin{equation}{section}
%Float package used to specify positioning
\usepackage{float}
%Depth of upto four in sections.
%\setcounter{secnumdepth}{3}
\usepackage{algorithmic}
\usepackage{algorithm}

%For hyper referencing
\usepackage[colorlinks={true},linkcolor={black},citecolor={black},urlcolor={black}]{hyperref}
%In case you use the package hyperref to create a PDF, the links to tables or figures will point to the caption of the table or figure, which is always below the table or figure itself. Therefore the table or figure will not be visible, if it is above the pointer and one has to scroll up in order to see it. If you want the link point to the top of the image you can use the package hypcap with:
\usepackage[all]{hypcap}

\usepackage[acronym]{glossaries}
\makeglossaries
\newacronym{al}{AL}{Artificial Life}
\newacronym{ca}{CA}{Cellular Automata}
\newacronym{cas}{CAS}{Complex Adaptive System}
\newacronym{ea}{EA}{Evolutionary Algorithm}
\newacronym{ec}{EC}{Evolutionary Computation}
\newacronym{ep}{EP}{Evolutionary Programming}
\newacronym{es}{ES}{Evolutionary Strategies}
\newacronym{formal}{FormAL}{Formal Artificial Life}
\newacronym{ga}{GA}{Genetic Algorithm}
\newacronym{gp}{GP}{Genetic Programming}
\newacronym{rna}{RNA}{Ribonucleic Acid}

\title{Modeling the Evolution of Mimicry}
\author{Mohiul Islam$^{1}$ \and Peter Grogono$^{1}$ \\
\mbox{}\\
$^1$Concordia University  \\
moh\_i@encs.concordia.ca \\
grogono@cse.concordia.ca}


\begin{document}
\maketitle

%A few paragraphs are required to bring an appropriate introduction.
Mimicry is a process of deception. It is an evolutionary process with the help of which organisms survive by deceiving its predator. But this deception happens only if the environment contains similar appearing noxious organisms which the predators find unpalatable. Palatable organisms mimic the unpalatable ones through the process of evolution for survival of its species. The objective of this paper is to present an agent based artificial life model for simulating this natural process of the evolution of mimicry.

Henry W. Bates first published in 1862 his findings about the similarities and dissimilarities between Heliconiinae and Ithomiinae butterflies. He found butterflies having similar appearance, exhibiting morphological features which point to completely different species even families. Even though Heliconiids are conspicuously colored, they are extremely abundant. They are also slow in mobility. Still predators in the surrounding area, mostly insectivorous birds do not prey on them, because of their inedible and unpalatable nature. Other edible and palatable species such as ithomiinae and pieridae, pretend to be heliconiids and thus enjoy protection. In general, the animal which is avoided by predator for unpalatable behavior is called the \textbf{model} and the imitating animal is called the \textbf{mimic}.

Bates was not able to explain some phenomena of mimicry. Occasionally two inedible unrelated butterfly species are amazingly similar in appearance. An explanation for this was provided by Fritz Muller in 1878. When there are multiple inedible species it is hard for predators to recognize each of them to know which one to consume and which one to avoid. Because of the predator's limited memory, all these species still lose their number even after being inedible. So to save this loss, and to prevent more sacrifice of their own kind, inedible species from different families also tend to evolve to have similar appearance. This phenomena is referred to as Mullerian Mimicry in the name of Fritz Muller. Like Batesian mimicry, Mullerian mimicry can evolve in two stages: the mutational, one way convergence stage followed by the gradual, mutual convergence stage.

Our model initializes with three kinds of agents. These agents have properties and behavior similar to the \textbf{model}, the \textbf{mimic} and the \textbf{predator}. We represent evolution of pattern for the model and the mimic with the help of \gls{ca}, as \gls{ca} can be easily represented by simple rules. Each predator is equipped with a Hopfield network, which gives it pattern recognition capability. The process of evolution occurs at the genetic level. 

This model has been designed to come up with efficient results and achieve the main objective, \textit{evolution of mimicry}. Creation and transformation of different mimicry ring and also the dynamics of it has been integrated to achieve interesting results. This model can also be considered as a complex adaptive system similar to Holland's work on Echo.

Data and analysis in this simulation has been concentrated on evaluating whether evolution of mimicry has taken place. This evaluation can be made with the number of different mimicry rings that has been created and the size of each of those rings along with the population of palatable and unpalatable species. Also it can be established whether Batesian Mimicry and Mullerian Mimicry have taken effect by analyzing the data set of these populations.

For all possible initial conditions, Batesian mimicry been found to take effect. From the population data sets it has been observed that for every ring of unpalatable species there is an existence of the palatable ring racing to reach the population count of its unpalatable counterpart. Effects of Mullerian mimicry can also be observed best for the experiment initialized with only unpalatable prey species. 

Analyzing the results tell us that we have successfully been able to simulate the evolution of mimicry. In addition, this model provides a more accurate simulation of the fascinating natural process of mimicry rings and their shift in population. It also verifies the explanation of the evolution of mimicry with punctuated equilibrium.
\printglossaries
\phantomsection \label{acronyms}
\end{document}
