%Concluding chapter
\chapter{Conclusion}
\label{chapter:conclusion}
If we consider the field of Artificial Life as a tool for biological inquiry then the model presented in this thesis was successful in terms of providing good insight in terms of understanding better the evolutionary process of mimicry as it happens in nature. As mentioned in section \ref{sec:result-conclusion} this model gives us verification of the theory of Turner in explaining the evolution of mimicry with punctuated equilibrium. Also it enforces findings from the work of Franks and Noble that multiple Mullerian mimics do not converge into one large ring.

If we consider Artificial Life as the study where we simulate natural processes to further extend capabilities in the field of computer science, then also this thesis is a success in making an appropriate emulation of the evolutionary process of mimicry. From the sections of \ref{sec:result-batesian-mimicry} and \ref{sec:result-mullerian-mimicry} we can conclude that the complex behavior of Batesian and Mullerian mimicry can be simulated with this model.

But considering Artificial Life as a subject where we essentially learn from nature, its behaviors, and use that knowledge to find better solution to existing problems in the field of computer science then this thesis has not been completely successful. We have not been able to find the appropriate problem solving scenario for which the natural process of mimicry can be applied.

Humans came to understand the concept of biological evolution and its significance when it was first proposed by Charles Darwin in 1859 in his book \textsl{On the Origin of Species} \cite{darwin1859}. Nearly 200 years later Alan Turing was the first who thought over creating machines which are capable of evolving itself (Section \ref{subsec:evo-comp-history}). Eventually in 1975 John Holland was able to make the most applicable use of the process of evolution to solve problems in computer science by his invention of Genetic Algorithms. The time for this invention was appropriate as the field of computer science was in its burgeoning state. Over the next couple of decades computers became one of most useful and important machinery in human history. Genetic Algorithms became popular as well because of increased capability of computers in term of speed and complexity.  

Even though we do not have any problem solving application for the evolutionary process of mimicry at this point, it does not imply that this study will not be useful in the future to give us solution to an important problem in computer science or any other field for that matter. The simulation presented in this thesis, itself, consists of building blocks of knowledge such as Hopfield Network and Cellular Automata. Maybe to use mimicry to solve a problem, other diverse building blocks are necessary which has not yet been invented.

The objective of any research is to look for \textsl{unexplored} territory to find new \textsl{knowledge}, \textsl{understanding} or \textsl{explanation}. From the research presented in this thesis we certainly get new \textsl{understanding} of the capability of computer science to emulate natures behavior. We also get to \textsl{explore} a new model for the evolutionary process of  mimicry. In addition to that, the model provides us with better \textsl{explanation} and \textsl{understanding} of nature, while enforcing some existing \textsl{knowledge} with greater support (section \ref{sec:result-mullerian-mimicry}). 