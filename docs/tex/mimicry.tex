%Chapter describing the mimetic process
\chapter{Mimicry}

\section{History}
Henry W. Bates first published in 1862 his findings about the similarities and dissimilarities between Heliconiinae and Ithomiinae butterflies, after 10 years of research in the Brazilian rain forest. For the next hundred years, it simulated heated discussion among all groups of people, scientists, philosophers, theologians, teachers and amateur naturalists. Bates collected ninety-four pieces of butterfly. He grouped them according to their similar appearance. He found butterflies having similar appearance, exhibiting morphological features which point to completely different species even families. Out of the ninety four species sixty seven are now classified as Ithomiinae, while twenty seven of them are Heliconiinae.

\section{Batesian Mimicry}
Even though Heliconiids are conspicuously colored, they are extremely abundant. They were also slow in mobility. Still predators in the surrounding area, mostly insectivorous birds do not prey on them, because of their inedible and unpalatable nature. Also because of this phenomenon other edible and palatable species such as ithomiinae and pieridae, pretend to be heliconiids and thus enjoy protection.

Repulsive animals, such as heliconiids are very conspicuously colored. Having this noticeable property, they are easily recalled by predators. Their wing pattern works as a warning to them. Once a predator has the knowledge of their inedible and unpalatable property, they would probably never attempt to try it again. As this is true, if any organism within close family and species, but being edible and having a deceptive resemblance to those conspicuously colored species will be avoided by the predators. Wickler \cite{wickler1986} expresses,
\begin{quote}
``Such unpalatable appearing and yet edible animals thus possess a false warning pattern, they 'act a part'. An actor is a mime, and so the representation of a false warning pattern was called \textit{mimicry}. Since Bates was the first to point out this phenomenon, it has received the name \textit{Batesian Mimicry} in his honor."
\end{quote}
In general, the animal which is avoided by predator for unpalatable behavior is called the \textbf{model} and the imitating animal is called the \textbf{mimic}.



\section{Mullerian Mimicry}
Bates was not able to explain some phenomenon of mimicry. Occasionally two inedible unrelated butterfly species are amazingly similar in appearance. An explanation for this was provided by Fritz Muller in 1878. Like Bates, Muller observed and caught butterflies in Brazil. When there are multiple inedible species it is hard for predators to recognize each of them to know which one to consume and which one to avoid. Because of predator's limited memory, all these species still loose their number even after being inedible. So to save this loss, and to prevent more sacrifice of their own kind, inedible species from different family also tend to evolve to have similar appearance. This phenomenon is referred to as Mullerian Mimicry in the name of Fritz Muller.

\section{Formation of Mimicry Rings}

\section{Other form of mimicry}
