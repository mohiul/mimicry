\begin{center}

\begin{Large}
\textbf{Abstract}\\[0.5cm]
\end{Large}

\begin{large}
\textbf{Modeling the Evolution of Mimicry}\\
\end{large}

\begin{small}
\textbf{Mohiul Islam}\\
\end{small}

\end{center}

A novel agent based, artificial life model, for the evolution of mimicry is presented. This model is a predator-prey co-evolution scenario where pattern representation phenotype is simulated with Cellular Automata, while behaviors of pattern recognition is configured with Hopfield Network. A visual three dimensional toroidal cube is used to construct a universe in which agents have complete freedom of mobility, genetic representation of behavior and reproduction capability to evolve new behaviors in successive generations. These agents are classified into categories of predator and prey species. Genome of prey species control their mobility and palatability, while 2D Cellular Automata (CA) is used to represent a pattern, where the rule to generate the CA is also genetically represented. Through evolution, successive generations of prey species have new pattern to represent them both visually and to the predators. Predators are agents with the primary purpose of providing selection pressure for the evolution of mimicry. They are equipped with Hopfiled Network memory to recognize new CA pattern and make intelligent decisions to consume the prey based on their level of palatability. Using the above construction of ideas, successful emulation of the natural process of mimicry is achieved. Also complex behavior pattern of Batesian and Mullerian mimicry is simulated and studied.

\setcounter{page}{3}
