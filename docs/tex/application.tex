\chapter{Application}

Part of this thesis was to think about whether the evolution of mimicry can be used in a problem solving scenario similar to evolutionary programming or genetic algorithm as it is used to optimize solutions using mathematical functions. Predator-prey co-evolution has also been used for solving interesting problems \cite{hillis1990}, giving better result than conventional methodologies. 

\paragraph{}
The biggest challenge faced while searching for the appropriate problem to be optimized is the idea of associating palatability with CA pattern with which the prey species has been represented. If we consider the CA pattern to be a certain solution to a problem then the set of 2D CA pattern among which the prey species vary could be considered as the solution set of the problem. And if we consider that the predator species are responsible for selecting the appropriate pattern based on their palatability, then we can consider palatability as the criteria for selection of the pattern (or the evaluation/fitness function for the case of Evolutionary Programming/Genetic Algorithm). 

\paragraph{}
If the predator associates the CA pattern with palatability then it does not make any sense of having a mimic in the solution space as mimics will have the same pattern with opposite palatability. So using the evolution of mimicry to solve an optimization problem is futile if we try conventional problems which already can be solved with the help of evolutionary programming or genetic algorithm. 
So the kind of problems we should be looking for applying the evolution of mimicry, are the ones which cannot be solved with conventional evolutionary methods. We should be looking for a problem where the idea of deception can be useful in term of solving it. Unfortunately during the time of this research we were unable to find such a problem. 
