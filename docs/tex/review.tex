%Review Chapter
\chapter{Review}

\begin{quote}
``The theory of evolution by cumulative natural selection is the only theory we know of that is in principle capable of explaining the existence of organized complexity" - Richard Dawkins \cite{dawkins1996}
\end{quote}

\section{Introduction}

\section{Evolutionary Computation}

``The fundamental metaphor of evolutionary computing relates this powerful natural evolution to a particular style of problem solving - that of trial and error" - Eiben \cite{eiben2003}

\subsection{History}

\paragraph{1948: Turing}
Ideas for evolutionary computation initiated from Alan Turing from his 1948 report titled ``Intelligent Machinery" where the expression of ``genetical or evolutionary search"  was used \cite{turing1948}. He suggested a range of ideas for systems which could be said to modify their own programs. 

\paragraph{1962: Bremermann}
Bremermann was the first to apply the concept of biological evolution to execute computer experiments for solving optimization problems \cite{bremermann1962}. He was the first to consider the problem of minimizing a real-valued fitness function. He selected very simple functions to optimize in order to provide a more tractable analysis of the evolutionary process.

\paragraph{1964: Rechenberg}
Rechenberg's contribution is considered mostly in providing Evolutionary Strategies. He invented a highly influential set of optimization method which were successfully applied to challenging problems such as aerodynamic wing design. \cite{rechenberg1973}

\paragraph{1965: L. Fogel, Owens and Walsh}
A seminal work that merged the fields of evolutionary computation and computational intelligence was by Fogel, Owens and Walsh in ``Artificial Intelligence through a simulation of evolution" \cite{fogel1966}. In this book, evolutionary process was applied to finite state automata to predict symbol strings generated from Markov processes and non-stationary time series. Evolutionary prediction of such kind was motivated by the concept that prediction is a keystone to intelligent behavior (defined in terms of adaptive behavior, in that the intelligent organism must anticipate events in order to adapt behavior in light of a goal). 

\paragraph{1975: Holland}
Perhaps the most significant work in terms of applications in the field of evolutionary computation comes from the inventor of genetic algorithms, John H. Holland. His ground breaking book ``Adaptation in natural and artificial systems" \cite{holland1975} is where he developed \textit{The Schema Theorem} which is the foundation for explanation of the power of Genetic Algorithms.

\paragraph{1992: Koza}
Koza's invention is the idea of Genetic Programming, which is an automated method for creating a working computer program from a high level statement of a problem. Starting from a high-level statement of ``what needs to be done" it searches through all possible permutation of steps and finds the optimum solution through evolutionary methods \cite{koza1992}.

Contemporary terminology denotes the fields of \textbf{Evolutionary Strategy}, \textbf{Evolutionary Programming}, \textbf{Genetic Algorithm} and \textbf{Genetic Programming} to be under one umbrella termed as \textbf{Evolutionary Computation} while the algorithms are called \textbf{Evolutionary Algorithms}.

\subsection{Evolutionary Algorithms}

\subsection{Evolutionary Strategies}

\subsection{Evolutionary Programming}

\subsection{Genetic Algorithm}

\subsection{Genetic Programming}

\section{Artificial Life}

\subsection{Complex Adaptive System}

\subsubsection{Echo}

\section{Cellular Automata}

\section{Conclusion}