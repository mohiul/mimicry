%Review Chapter
\chapter{Review}

\begin{quote}
\textsl{``The theory of evolution by cumulative natural selection is the only theory we know of that is in principle capable of explaining the existence of organized complexity" - Richard Dawkins \cite{dawkins1996}}
\end{quote}

\section{Introduction}

\section{Evolutionary Computation}

\textsl{``The fundamental metaphor of evolutionary computing relates this powerful natural evolution to a particular style of problem solving - that of trial and error" - Eiben \cite{eiben2003}}

\subsection{History}
\label{subsec:evo-comp-history}

\paragraph{Turing(1948)}
Ideas for evolutionary computation initiated from Alan Turing from his 1948 report titled \textsl{``Intelligent Machinery"} where the expression of \textsl{``genetical or evolutionary search"}  was used \cite{turing1948}. He suggested a range of ideas for systems which could be said to modify their own programs. 

\paragraph{Bremermann(1962)}
Bremermann was the first to apply the concept of biological evolution to execute computer experiments for solving optimization problems \cite{bremermann1962}. He was the first to consider the problem of minimizing a real-valued fitness function. He selected very simple functions to optimize in order to provide a more tractable analysis of the evolutionary process.

\paragraph{Rechenberg(1964)}
Rechenberg's contribution is considered mostly in providing Evolutionary Strategies. He invented a highly influential set of optimization method which were successfully applied to challenging problems such as aerodynamic wing design. \cite{rechenberg1973}

\paragraph{L. Fogel, Owens and Walsh (1965)}
A seminal work that merged the fields of evolutionary computation and computational intelligence was by Fogel, Owens and Walsh in \textsl{``Artificial Intelligence through a simulation of evolution"} \cite{fogel1966}. In this book, evolutionary process was applied to finite state automata to predict symbol strings generated from Markov processes and non-stationary time series. Evolutionary prediction of such kind was motivated by the concept that prediction is a keystone to intelligent behavior (defined in terms of adaptive behavior, in that the intelligent organism must anticipate events in order to adapt behavior in light of a goal). 

\paragraph{Holland(1975)}
Perhaps the most significant work in terms of applications in the field of evolutionary computation comes from the inventor of genetic algorithms, John H. Holland. His ground breaking book \textsl{``Adaptation in natural and artificial systems"} \cite{holland1975} is where he developed \textit{The Schema Theorem} which is the foundation for explanation of the power of Genetic Algorithms.

\paragraph{Koza(1992)}
Koza's invention is the idea of Genetic Programming, which is an automated method for creating a working computer program from a high level statement of a problem. Starting from a high-level statement of \textsl{``what needs to be done"} it searches through all possible permutation of steps and finds the optimum solution through evolutionary methods \cite{koza1992}.

Contemporary terminology denotes the fields of \textbf{Evolutionary Strategy}, \textbf{Evolutionary Programming}, \textbf{Genetic Algorithm} and \textbf{Genetic Programming} to be under one umbrella termed as \textbf{Evolutionary Computation} while the algorithms are called \textbf{Evolutionary Algorithms}.

\subsection{Evolutionary Algorithms}
Algorithms which are inspired by the strategy of \textit{survival of the fittest} from evolutionary biology, and uses different methods to optimize mathematical expressions which are used to evaluate a pool of population of solutions over generation, can be defined as evolutionary algorithms. There are many different forms of evolutionary algorithm. All of which can be generalized to follow their pattern from natural selection. The mathematical expression, also termed as \textbf{fitness function} is used to evaluate the fitness of each individual from the set of existing solution space. An individual or a solution which provides higher output from the fitness function is considered as more fit to survive in the environment. So given a problem the algorithm initializes the environment with a random set of population or solutions. Then it uses the fitness function to select a better set of solutions from the existing once. After that different operators of \textbf{mutation} and \textbf{recombination} are applied to the selected set of solution to come up with a new generation of solution set born from the existing ones. Again the fitness function is applied to the current set of population to come up with a better set and this process iterates until the evaluation from the fitness function is satisfactory to give the best set of solutions.

According to Eiben \cite{eiben2003} this process has two fundamental forces that form the basis of evolutionary systems:

\begin{itemize}
	\item \textsl{``Variation operators (recombination and mutation) creates the necessary diversity and thereby facilitate novelty."}
	\item \textsl{``Selection acts as a force pushing quality."}
\end{itemize}

Important components of evolutionary algorithms are

\paragraph{Representation (Individuals or Solution Space)}
Evolutionary Algorithms(EA) are considered as robust problem solvers as it provides evenly good performance over a wide range of problems. To apply evolutionary algorithm to this wide range the most important part is representation of the solution space. In EA representation is analogous to the complex pathway that exits between \textbf{genotype} and \textbf{phenotype} of a biological organism.

\paragraph{Evaluation Function (Fitness Function)}
This is the function with which the solution is evaluated. The outcome of applying evolutionary algorithm to the problem depends on this function. It is defined to evaluate each of the solutions of the problem. In biology evaluation function is analogous to the decision of survival of any species in an environment. 

\paragraph{Population}
The population is the solution set of the problem over which EA will be applied. It is literally analogous to the same meaning in biology. The diversity of the population is very important. If the initial random population of solutions are diverse then there is more possibility of reaching an optimal solution space very quickly, and it also avoids reaching local optimum instead of the global optimum.

\paragraph{Variation operators, recombination and mutation}
Considering arity (number of object as input) variation operators are divided into two kinds. The mutation operator accepts only one operand. The process of mutation is similar to the way it happens in biology. It is a stochastic process, meaning, its output the child depends on the outcomes of a series of random choices. The operator is responsible for causing a random unbiased change. Mutation has very different roles in different fields of EC, for example in Genetic Programming it is not used at all, while in genetic algorithms it is a very pivotal process as it provides fresh blood to the existing set of solutions. Getting a new mutated child is like stepping into a new solution space outside of the ones which already exist in the parent pool. A recombination operator is responsible for merging two parent genotypes into one or two offspring genotypes. It is also a stochastic process depending on the choice of the parts of the parent individuals. Recombination operators with high arity (using more than two parent) are mathematically possible and easy to implement but does not have any biological equivalent. According to Eiben \cite{eiben2003} \textsl{``the principle behind recombination is simple - by mating two individuals with different but desirable features, we can produce an offspring that combines both of those features."} It is important that variation operators are defined based on the representation of the problem under consideration. 

\subsection{Genetic Algorithm}
Genetic Algorithms(GA) have the most wide set of applications. It was first conceived by John H. Holland for studying adaptive behavior as in \textit{Adaptation in natural and artificial systems} \cite{holland1975}. They are mainly considered as function optimizers. In general GA have many diverse representation schemes considering the solution set of the problem. Depending on the problem scenario an appropriate representation could be Binary, Grey Code, Integer, Real-Valued, Floating-point or Permutation based. 

Using GA for a binary representation mutation can be single point or multiple point, where a random location is selected from the binary genome and the bit is flipped. For integer representation random resetting is used instead of bit flipping. Creep mutation is another scheme which was designed for ordinal attributes and works by adding a small (positive or negative) value to each gene with a random probability. For a real-valued or floating-point mutation none of the above strategy is applicable. As for this case mutation is applied to a random gene maintaining a certain range of value within an upper and a lower bound and using a standard for randomization, such as Uniform, Gaussian or Cauchy distribution. For permutation representation, swap mutation is a strategy used by simply swapping the location of two randomly selected genes. Insert mutation is another strategy where a random gene is selected and inserted to a random location in the genome, causing the other genes to move a single space. In scramble mutation, location of a collection of randomly selected genes are scrambled, while in inversion mutation serialized location of a set of genes are inverted. 

GA also have many different recombination operators for each representation. For binary and integer representation one-point crossover and \textit{N}-point-crossover are the most frequently used. These two operators work by dividing the parent genome at one or multiple random points and combining them to create an offspring. Uniform-crossover \cite{sywerda1989} is another method which works by treating each gene independently and making a random choice at to which parent it should inherit from. Recombination for floating-point representation can use the same operators like binary and integer, and when it does it is termed as discreet recombination. Another type is intermediate or arithmetic recombination where the offspring do not just receive a part of their parent genome but instead their genomes are derived from their parent using an arithmetic formula. Permutation representation has many interesting recombination operators, out of which partially mapped crossover was first proposed by Goldsberg and Lingle \cite{goldberg1985} for the traveling salesman problem and since then it has become one of the most widely used operators for adjacency type problems. Other useful operators for permutation representation are Edge crossover, Order crossover and Cycle crossover, where their name provides more or less an appropriate description of their operation.

In regards to parent selection the most commonly used process by GA is fitness proportion selection, where the selection probability of a parent depends on the absolute fitness value of the individual compared to the absolute fitness values of the rest of the population. A fitness value is evaluated from the evaluation or the fitness function. Introduced by Holland \cite{holland1975} this procedure has a few problems. Premature convergence being one where individuals that are a lot better than the rest take over the entire population. Also when fitness values are all very close there is almost no selection pressure in evolution of the fittest. Inspired by the observed drawbacks of fitness proportion selection a method was proposed by Baker \cite{baker1987} termed as Ranking selection. This procedure preserves a constant selection pressure by sorting the population on the basis on fitness, and then allocating selection probabilities to individuals according to their rank. Both of the above mentioned selection procedures requires knowledge of the entire population. But when the population size is very large Tournament selection is an effective process. In this case a randomized pre-selection is done over the large population and then they are ranked according to their fitness value and selected as parent. 

Genetic Algorithms are one of the most effective strategy used in the field of evolutionary computation to solve optimization problems. Then again there are limitations. When the solution set of a problem do not have a gnomic representation, genetic algorithms cannot play any role in improving it. So the set of optimization problems to which GA can be applied is not universal.

\subsection{Evolutionary Strategies}
The key important property of the set of algorithms termed as ``Evolutionary Strategy" is \textbf{self-adaptation} of strategy parameters. The  parameters of Evolutionary Algorithm are included in the chromosomes and they co-evolve with the solution set while running the algorithm. To summarize these set of algorithms, they usually have real-valued vectors, their recombination process is discrete or intermediary, their mutation process uses Gaussian perturbation while parent selection is uniform random. In terms of their attribute feature they are fast and good optimizer for real-valued optimization.

\subsection{Evolutionary Programming}
The primary objective of evolutionary programming was to generate artificial intelligence by simulating evolution as a learning process \cite{fogel1966}. Capability of a system to adapt its behavior to meet some specified goal in the environment was considered as intelligence. Fogel used finite state machines as predictors and evolved them. Usually representation for evolutionary programming is with real-valued vectors. Its parent selection is deterministic, where each parent creates one offspring via mutation. There is no recombination operator for evolutionary programming. Mutation is applied with Gaussian perturbation. While survivor selection is probabilistic. 

\subsection{Genetic Programming}
Genetic programming is typically applied to machine learning tasks such as prediction and classification. Its attribute feature behaves similar to neural network and usually has huge populations of solution combination. The algorithm itself is a very slow process. These sets of algorithms use non-linear chromosomes such as trees and graphs. Mutation is mostly avoided for genetic programming. Recombination operators work by exchanging subtrees. Mutation if applied will be a random changing process in the tree. Parent selection process is usually fitness proportional while the survivor selection process is usually replacement of generations. From a technical point of view, genetic programming is simply a variant of genetic algorithm working with a different data structure: the chromosomes are trees. 

\section{Artificial Life}
As Artificial Intelligence is the field of computer science where scientists take concepts from psychology and implement into computer science. In contrast to that, in Artificial Life, scientist take concepts from Biology, specially Evolutionary Biology and genetics and implement into computer science. 

Christopher G. Langton is considered as one of the founders of the field of Artificial Life. He invented the term in the late 1980's while organizing the first `\textsl{International Conference on the Synthesis and Simulation of Living Systems'} at the Los Alamos National Laboratory in 1987.

In defining Artificial Life Christopher G. Langton states the following,

\begin{quote}
\textsl{`` `Art' + `Life' = Artificial Life: Life made by Man rather than by Nature. Our technological capabilities have brought us to the point where we are on the verge of creating `living' artifacts. The field of Artificial Life is devoted to studying the scientific, technological, artistic, philosophical, and social implications of such an accomplishment."}
\end{quote}

The field of Artificial Life (AL) has been described by Taylor also as a tool for biological inquiry \cite{taylor1993}. While providing a brief survey over different AL models he talks about \textbf{Wetware systems} which work at the molecular level, the \textbf{Software systems} which work at the cellular level and the \textbf{Hardware systems} which works at the organism level.

%Describe Wetware, Software and Hardware systems of artificial life. 

%\section{History}

%\paragraph{Jon Von Neumann}

%\paragraph{Tierra}

%\paragraph{Avida}

\subsection{Complex Adaptive System}
\label{subsec:complex-adaptive-system}
Complex adaptive systems (CAS) are special cases of complex systems. They are complex as they are diverse and made up of multiple interconnected elements and adaptive as they have the capacity to change and learn from experience. Examples of complex adaptive systems include the stock market, social insect and ant colonies, the biosphere and the ecosystem, the brain and the immune system, the cell and the developing embryo, manufacturing businesses and any human social group-based endeavor in a cultural and social system such as political parties or communities. This includes some large-scale online systems, such as collaborative tagging or social bookmarking systems.

The following definition of Complex Adaptive System is comprehensive.

\begin{quote}
\textsl{``A Complex Adaptive System (CAS) is a dynamic network of many agents (which may represent cells, species, individuals, firms, nations) acting in parallel, constantly acting and reacting to what the other agents are doing. The control of a CAS tends to be highly dispersed and decentralized. If there is to be any coherent behavior in the system, it has to arise from competition and cooperation among the agents themselves. The overall behavior of the system is the result of a huge number of decisions made every moment by many individual agents." - John H. Holland}
\end{quote}

Holland also provides some common kernel of similarities and dissimilarities between different complex adaptive systems in \textsl{Adaptation in Natural and Artificial Systems} \cite{holland1975}:

\begin{enumerate}
	\item \textsl{``All complex adaptive system involves large number of parts undergoing a kaleidoscopic array of simultaneous non linear interactions."}
	\item \textsl{``The impact of these systems in human affairs centers on the aggregate behavior, the behavior of the whole."}
	\item \textsl{``The interactions evolve over time, as the parts adapt in an attempt to survive in the environment provided by the other parts."}
	\item \textsl{``Complex adaptive systems anticipate."}
\end{enumerate}

\subsection{Echo}
Echo is a class of simulation model of a complex adaptive system, designed primarily for gedanken experiments rather than a precise simulation. Echo provides, for the study of populations of evolving, reproducing agents distributed over a geography, with different inputs of renewable resources at various sites. Each agent has simple capabilities - offense, defense, trading and mate selection - defined by a set of chromosomes. Even though these capabilities are simple and defined simply, they provide a rich set of variations illustrating the four kernel properties of complex adaptive systems described in section \ref{subsec:complex-adaptive-system}.

\subsubsection{Criteria}
Echo has been constructed on a set of criterion. The following is a short summary on it extracted from Holland's, \textsl{Hidden Order: How Adaptation Builds Complexity} \cite{holland1996}.

\begin{itemize}
	\item Simplicity is considered as the first and foremost criteria. Instead of emulating a real system Echo is more of a thought experiment. Simplicity is attained by carefully constraining agent interaction and also giving the agents only a primitive internal model. 
	\item Actions and interaction of each agent in echo has been designed to be part of a wide range of \textsl{CAS} setting. Their mobility is also part of a wide range of geographical setting. Input to the environment which is considered as a stimuli or resource, can also be diversified.
	\item Fitness of each of agent, is an evolving criteria. Instead of considering fitness as an external exogenous factor, it should be dependent on the context provided by the site and other agents. The fitness value, instead of being a constant should evolve along with the system. 
	\item ``The primitive mechanisms in Echo should have ready counter parts in all \textsl{CAS}". Interpretation of results from the Echo model should be direct and ready-made. As simulation is nothing but manipulation of numbers and symbols, it is always possible to make fanciful and deceiving interpretation of them, taking the opportunity to prove one's point. So by applying primitive mechanism for interpretation it is possible to avoid misinterpretation. Another advantage is with the help of interpretations, selected mechanisms can be shown to be sufficient to generate the phenomenon of interest. Even though simulation cannot establish that a given mechanism is actually present, but it can establish its sufficiency or plausibility.
	\item The architecture of the Echo model should be flexible enough so that other well-studied mathematical models of particular \textsl{CAS} can be incorporated whenever required. Among other models the significant ones include, Dawkins' \textsl{biological arms race} \cite{dawkins1990} and Brower's \textsl{survival of mimics} \cite{brower1988} from ecology; Wicksell's \textsl{Triangle} \cite{marimon1989} and \textsl{overlapping generation models} \cite{anderson1989} in economics; the \textsl{prisoner's dilemma game} \cite{axelrod1984} in political science; Holland's \textsl{two-Armed bandits} \cite{holland1975} in operational research and Perelson's \textsl{antigen-body matching} \cite{perelson1999} in immunology. 
	\item As mathematical analysis provides the surest route to arrive at a valid generalization point, most aspects of the Echo model should be amenable to mathematical analysis. 
\end{itemize}

In developing Echo using the above set of criterion, a step by step approach has been taken. Among these steps of improvement, the work by Hraber titled ``The Ecology of Echo" \cite{hraber1997} is more from an ecological perspective and has been discussed in the following.

\subsubsection{Organization}
As Echo is an agent-based model, it has a collection of agents distributed in a two dimensional site which contains resource. Agents in Echo have freedom and flexibility to move around this environment. These agents are capable of replicating themselves, while also having the capability of making complex interactions. Interactions include combat, trading and mating.

\paragraph{Resources and Sites}
Resource in a site of Echo is its very foundation. They are represented by letters (a,b,c,d). Everything in Echo is constructed by combining these letters as strings. These resources are treated much like an `atom' and by combining them as a molecular string, agents are constructed. Geographically, Echo is a collection of interconnected sites. Each site contains a fountain which is the source of resource for the agents in Echo. The amount of resource available in each site varies in time. A site with no resource can be compared to a `desert' while another site with increased availability of letters can be a water spring or a pond. 

\paragraph{Agents}
`Agents have a genome that is roughly analogous to a single chromosome in an haploid species'. An agents \textsl{tag} is an external phenotype of its respective genotype (such as color). Agents have internal conditions or rules of interaction, which are genetically represented and they encode the agents internal preferences, behavioral rules and so forth. So interactions between multiple agents occur based on their conditions (rules for interactions) and tags (appearance). This results in sophisticated interaction between agents such as deception (mimicry, bluffing) and in-transitivities. `The six (external) tag and (internal) condition genes possessed by every agent are the offense tag, defense tag, mating tag, combat condition, trade condition and mating condition'. These genes are used to determine what sort of interaction between agents are will take place and also the outcome of their interaction. 

Agents have two ways of reproduction, one being self replication, while the other is by interaction between a pair of agents. 

\paragraph{Replication}
The replication process for agents is an asexual reproduction. An agent needs to collect sufficient resource to start the self-replication process. After replication, the newly created agent receives a fraction of its parent's resource. Spontaneous mutation is applied to the gene for this process, where different mutation rate is applied to different parts of the gene. 

\paragraph{Interactions}
Their are diverse interactions between agents consisting of combat, trading and mating. Agents need to be in the same location in site for any interaction to take place. Testing for which interaction to take place is always conducted in the order of combat, trade and mating. 

\paragraph{Combat}
The interaction of combat is analogous to real world antagonistic behavior, where one agent is killed resulting in the survivor agent to receive all its resource. Depending on each agent's offense and defense tag and the tag prefixes, the winner is decided. A \textsl{combat matrix} is also calculated with the genome of individual agents, which helps in calculating the winner.

\paragraph{Trading}
When two agents are chosen to interact and they do not engage in combat, they are given the opportunity to trade and mate. Trading and mating are very much a mutual agreement between agents compared to combat, which is more of a destructive process. ``Agents A and B will trade if A's trading condition is a prefix of B's offense tag and vice versa". Trading takes place for excess resources between agents, where excess is defined as the amount which the agent possess above the amount required for a replication process, plus some reserves. The value of the reserve is controlled with the help of a simulation parameter. A deceptive process is introduced here while trading takes place, when an agent might not have excess resource but will still be engaged in trade, as it is not possible for the other agent to know the resource of its opposite interactor. 

\paragraph{Mating}
Only if two agents find each other suitable, they will engage in the mating process. Suitability for mating is decided on certain conditions based on the mating tag of individual agents. A form of two point crossover takes place to perform the mating action and produce diverse new agents. 

\subsubsection{Discussion of Results}
The Echo model has good resemblance with natural ecosystem. It has stable patterns of interactions among genetic variants, or tropic networks. It consists of different levels of resources along with relative species abundance. Even different cataclysmic events such as meteor, drought and so fourth are introduced. Also several isolation effects are observed where a number of sites are allowed to evolve independently. In addition to that Echo contains interesting flow of resources. 

Two separate experiments are done with the above simulation to observe whether species abundance pattern can be used to find ecologically plausible behavior in Echo. To make a falsifiable comparison with biological systems, two ecological patterns are considered, each of which has similar quantitative properties in many different ecosystems that exist in nature. The objective is to compare these naturally occurring ecosystems with Echo. The ecological patterns are the Preston  curve and the species-area scaling relation. This allowed comparison of Echo diversity patterns with empirical patterns of species abundance.

Preston's canonical log normal distribution is the most widely accepted formalization of the relative commonness and rarity of species. The species-area scaling relation is also an important ecological pattern. The theory of island biography is predicted on species-area relations. In conservation biology, this relation has been used to predict the effects of different reserve size, with species diversity. Although the species-area relation can be derived from Preston's canonical log-normal distribution, a satisfactory explanation of the processes that regulate this relation has not been advanced. 

Now the question is, does Echo exhibit the robust ecological phenomenon of Preston curve and species area scaling relation? For Preston's curve plot, the data show, qualitatively the pattern of genome abundances in Echo population resembles the general patterns found in biological systems, although some differences are observed. But for species area curve some discrepancy was observed between the expected result and the observed values. Several factors were raised for this discrepancy. One was the choice of the simulation parameter. Second being the definition of different species in Echo as agents having different genome is considered from different species. Another factor being the balance between rates of speciation and extinction of population. 

Finally to answer the above question, it has been concluded that, Echo showed good quantitative agreement with naturally occurring species abundance distribution and species area scaling relation.

\section{Cellular Automata}

\section{Conclusion}