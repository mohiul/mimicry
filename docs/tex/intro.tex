%Introductory Chapter
\chapter{Introduction}
\label{chapter:introduction}
\pagenumbering{arabic}

%A few paragraphs are required to bring an appropriate introduction.
Mimicry is a process of deception. It is an evolutionary process with the help of which organisms survive by deceiving its predator. But this deception happens only if the environment contains similar appearing noxious organisms which the predators find unpalatable. Palatable organisms mimic the unpalatable ones through the process of evolution for survival of its species.

Artificial Life is an interdisciplinary study of life and life-like processes that uses synthetic methodology. This thesis consists of such a work which concentrates specifically on the life-like process of the evolution of mimicry. The synthetic tools used for this simulation is a three dimensional graphical environment, Cellular Automata and Neural Network. 

%Individual chapter descriptions begins from chapter 2
Chapter \ref{chapter:review} is intended to provide background knowledge upon which this thesis is based. It reviews different fields of science where the concept of evolution has been applied to solve complex computer science problems, and achieved better result than other conventional methods. It also discusses the field of Artificial Life which is an inter disciplinary field of science where scientist take concepts from biology, economics and physics to build complex and adaptive models to emulate nature and its life like behavior.

Chapter \ref{chapter:mimicry} discusses an inspiration from the field of evolutionary biology which is the primary purpose of building the model of this thesis. Mimicry is an evolutionary process, which not only provides us with insight into the process of evolution as it happens in nature but also is fascinating to study and learn how nature adapts itself in complicated situation for the survival of its species. In addition, this chapter discusses the different kinds of mimicry such as Batesian and Mullerian, while including how the concept of mimicry ring helps us explain each of these cases. 

Chapter \ref{chapter:model} contains description of an agent based artificial life model, which is the primary purpose of this thesis. Inspired by the evolutionary mimetic process we present a model whose purpose is not only to emulate the evolution of mimicry but also to come up with complicated scenario of Batesian and Mullerian mimicry with the help of mimicry ring. This model is based on a FormAL (Formal Artificial Life) framework which is similar to Holland's Echo model but with a slightly different approach. 

Chapter \ref{chapter:results} deals with the results extracted from the model and its analysis to evaluate successful emulation of mimicry. The results are extracted by applying different initial conditions to the model. As this work is with complex adaptive system, the data that we extract from such system is also expected to be convoluted. But our effort in this section is to come up with simplified analysis of the data to be able to comprehend complex activities in the model. 

Chapter \ref{chapter:application} tries to provide direction for possible future work. The goal of this thesis was not only to come up with a model that emulate natures mimetic process but also to apply this method to solve problems in the field of computer science. This approach is similar to the way evolutionary algorithms and genetic programming help us in optimizing solutions of complicated problems. 

Chapter \ref{chapter:conclusion} is an effort to bring a conclusion to justify the work done on this thesis. 