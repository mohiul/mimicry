\documentclass[letterpaper]{article}
\usepackage{natbib,alifeconf}
%Graphics package
\usepackage{graphicx}
\DeclareGraphicsExtensions{.png,.jpg,.pdf}
%For multirow feature to draw tables
\usepackage{multirow}
%Middle allignment in tables.
\usepackage{array}
%For Mathematics
\usepackage{amsmath}
\numberwithin{equation}{section}
%Float package used to specify positioning
\usepackage{float}
%Depth of upto four in sections.
%\setcounter{secnumdepth}{3}
\usepackage{algorithmic}
\usepackage{algorithm}

%For hyper referencing
\usepackage[colorlinks={true},linkcolor={black},citecolor={black},urlcolor={black}]{hyperref}
%In case you use the package hyperref to create a PDF, the links to tables or figures will point to the caption of the table or figure, which is always below the table or figure itself. Therefore the table or figure will not be visible, if it is above the pointer and one has to scroll up in order to see it. If you want the link point to the top of the image you can use the package hypcap with:
\usepackage[all]{hypcap}

\usepackage[acronym]{glossaries}
\makeglossaries
\newacronym{al}{AL}{Artificial Life}
\newacronym{ca}{CA}{Cellular Automata}
\newacronym{cas}{CAS}{Complex Adaptive System}
\newacronym{ea}{EA}{Evolutionary Algorithm}
\newacronym{ec}{EC}{Evolutionary Computation}
\newacronym{ep}{EP}{Evolutionary Programming}
\newacronym{es}{ES}{Evolutionary Strategies}
\newacronym{formal}{FormAL}{Formal Artificial Life}
\newacronym{ga}{GA}{Genetic Algorithm}
\newacronym{gp}{GP}{Genetic Programming}
\newacronym{rna}{RNA}{Ribonucleic Acid}

\title{Modeling the Evolution of Mimicry}
\author{Mohiul Islam$^{1}$ \and Peter Grogono$^{1}$ \\
\mbox{}\\
$^1$Concordia University  \\
moh\_i@encs.concordia.ca \\
grogono@cse.concordia.ca}


\begin{document}
\maketitle

\begin{abstract}
A novel agent based, artificial life model, for the evolution of mimicry is presented. This model is a predator-prey co-evolution scenario where pattern representation phenotype is simulated with Cellular Automata (CA), while behaviors of pattern recognition is configured with Hopfield Network. A visual three dimensional toroidal cube is used to construct a universe in which agents have complete freedom of mobility, genetic representation of behavior and reproduction capability to evolve new behaviors in successive generations. These agents are classified into categories of predator and prey species. Genome of prey species control their mobility and palatability, while 2D CA is used to represent a pattern, where the rule to generate the CA is also genetically represented. Through evolution, successive generations of prey species develop new patterns to represent them both visually and to the predators. Predators are agents with the primary purpose of providing selection pressure for the evolution of mimicry. They are equipped with Hopfield Network memory to recognize new CA pattern and make intelligent decisions to consume the prey based on their level of palatability. Using the above construction of ideas, successful emulation of the natural process of mimicry is achieved. Also complex behavior pattern of Batesian and Mullerian mimicry is simulated and studied.
\end{abstract}

%Introductory Section
\section{Introduction}
\label{section:introduction}

%A few paragraphs are required to bring an appropriate introduction.
Mimicry is a process of deception. It is an evolutionary process with the help of which organisms survive by deceiving its predator. But this deception happens only if the environment contains similar appearing noxious organisms which the predators find unpalatable. Palatable organisms mimic the unpalatable ones through the process of evolution for survival of its species. The objective of this paper is to present an agent based artificial life model for simulating this natural process of the evolution of mimicry.

According to Langton, Artificial Life is `\textsl{Life made by Man rather than by Nature'}. Taylor also defines it as a tool for biological inquiry \citep{taylor1993}. While providing a brief survey over different AL models he talks about \textsl{Wetware systems} which work at the molecular level, \textsl{Software systems} which work at the cellular level and \textsl{Hardware systems} which works at the organism level. The initial contribution of \textsl{software systems} in artificial life was from John von Neumann when he designed the first artificial-life model (without referring to it as such), the famous self-reproducing, computation-universal cellular automata \citep{neumann1966}. He tried to understand the fundamental properties of living systems, especially self-reproduction and the evolution of complex adaptive structures, by constructing simple formal systems that exhibit those properties.

Being a special case of complex systems, \gls{cas} are diverse and are made up of multiple interconnected elements, and adaptive as they have the capacity to change and learn from experience. Echo \citep{hraber1997} is a class of simulation model of \gls{cas}, providing a population of evolving, reproducing agents distributed over a geography, with different inputs of renewable resources at various sites. Each agent has simple capabilities: offense, defense, trading and mate selection, defined by a set of chromosomes. Even though these capabilities are defined simply, they provide a rich set of variations illustrating the four kernel properties of \gls{cas} described by Holland \citep{holland1996}.

\section{The Inspiration: Mimicry}
\label{section:mimicry}

Henry W. Bates first published in 1862 his findings about the similarities and dissimilarities between Heliconiinae and Ithomiinae butterflies, after 10 years of research in the Brazilian rain forest. Bates collected ninety-four pieces of butterfly. He grouped them according to their similar appearance. He found butterflies having similar appearance, exhibiting morphological features which point to completely different species even families. Out of the ninety four species, sixty seven are now classified as Ithomiinae, while twenty seven of them are Heliconiinae.

\subsection{Batesian Mimicry}
Even though Heliconiids are conspicuously colored, they are extremely abundant. They are also slow in mobility. Still predators in the surrounding area, mostly insectivorous birds do not prey on them, because of their inedible and unpalatable nature. Also because of this phenomenon other edible and palatable species such as ithomiinae and pieridae, pretend to be heliconiids and thus enjoy protection.

Repulsive animals, such as heliconiids are very conspicuously colored. Having this noticeable property, they are easily recalled by predators. Their wing pattern works as a warning to predators. Once a predator has the knowledge of their inedible and unpalatable property, they would probably never attempt to try it again. As this is true, if any organism within close family and species, but being edible and having a deceptive resemblance to those conspicuously colored species will be avoided by the predators. 

In general, the animal which is avoided by predator for unpalatable behavior is called the \textbf{model} and the imitating animal is called the \textbf{mimic}.

\subsection{Mullerian Mimicry}
\label{subsec:mullerian-mimicry}
Bates was not able to explain some phenomena of mimicry. Occasionally two inedible unrelated butterfly species are amazingly similar in appearance. An explanation for this was provided by Fritz Muller in 1878. When there are multiple inedible species it is hard for predators to recognize each of them to know which one to consume and which one to avoid. Because of predator's limited memory, all these species still lose their number even after being inedible. So to save this loss, and to prevent more sacrifice of their own kind, inedible species from different family also tend to evolve to have similar appearance. This phenomena is referred to as Mullerian Mimicry in the name of Fritz Muller.

\subsection{Evolutionary Dynamics of Mimicry}
\label{subsec:evolutionary-dynamics-of-mimicry}
The dynamics of mimicry has been investigated by Turner \citep{turner1988}, where he states that the evolution of mimicry can be explained best by the process of punctuated equilibrium instead of phyletic gradualism. He came up with a synthetic theory \citep{turner1988}, which was originated by Poulton \citep{poulton1912} and Nicholson \citep{nicholson1927}, termed as the \textbf{two stage model}. This theory states that mimicry normally arises in two steps. A comparative large mutation achieves a good approximate resemblance to the model; it is followed by gradual evolutionary changes that refine the resemblance, in many cases to a high degree of perfection \citep{sheppard1972} \citep{ford1964}. This two-stage theory has been applied for the explanation of Mullerian mimicry as well.

\subsubsection{Mimicry Ring}
Any theory of Mullerian mimicry has to take into account the phenomenon of the coexistence of multiple mimicry rings. If we examine the local butterfly fauna in any area of the world, we will find that between all the aposomatic (warningly colored and defended) species present there are normally only a limited number of different patterns, normally far smaller than the number of species. Each cluster of species, all sharing a common pattern, is termed as Mullerian mimicry ring. Thus, in the rain forest of South and Central America, most of the long-winged butterflies (ithomiids, danaids, and heliconids) belong to one of only five different rings.

Like Batesian mimicry, Mullerian mimicry can evolve in two stages: the mutational, one way convergence stage followed by the gradual, mutual convergence stage. It is worth mentioning that in the first stage only the less protected species can adopt the pattern of the better protected species; mutations in the other direction is not favored.

%Chapter describing the model
\section{The Model: Evolution of Mimicry}
\label{section:model}

Our model initializes with three kinds of agents. These agents have properties and behavior similar to the \textbf{model}, the \textbf{mimic} and the \textbf{predator}. We represent evolution of pattern for the model and the mimic with the help of \gls{ca} \citep{Wolfram2002}. \gls{ca} can be easily represented by simple rules, which can be expressed as a binary string. The predator will be equipped with a Hopfield network \citep{hopfield1982}, to have pattern recognition capability. The process of evolution will be occurring at the genetic level. 

The choice of Hopfield Network memory for a predator can be considered appropriate as the number of patterns which can be recognized by this network is inversely proportional to the accuracy of recall. As more patterns are memorized, Hopfield network tends to make more errors. This behavior will be appropriate for the simulation of Mullerian mimicry. Mullerian mimicry happens because of limited memory of the predators. Because of this limited memory, multiple inedible butterflies seems to converge to a single ring.

Similar to the \textsl{Laws and Life} project by Peter Grogono \citep{grogono2003} the environment is designed as three dimensional, while the space will be of toroidal nature.

\subsection{Past Work}
Various models of mimicry has been simulated and explored. The model by Turner \citep{turner1996} and the mathematical model of Huheey \citep{huheey1988} tend to focus on the selective pressure on prey brought about by the particular learning abilities of the predator, and employ simple Monte Carlo or mathematical approaches.

Sherratt \citep{sherratt2002} provides an innovative perspective on the evolution of warning signals by considering co-evolving predator and prey populations. The model's predators are deterministic, in that they have a fixed behavioral strategy over their lifetime, and cannot learn from experience. For both cryptic and conspicuous prey, each predator has fixed policy of either attacking or avoiding.

\subsubsection{Models by Franks and Noble}
\label{subsubsec:models-by-frank-and-noble}
The latest work on modeling evolution of warning signals and mimicry with individual based simulation is done by Franks and Noble. Their initial work \citep{franks2002} seems to focus on putting some conditions of mimetic evolution in an individual based model with multiple species preyed upon by a single abstract predator, where the appearance of each prey species can evolve but their palatability is fixed.

On 2003 \citep{franks2003} another model for the origin of mimicry ring has been proposed which is based on two working hypothesis:

\begin{enumerate}
	\item \textsl{All of the Mullerian mimics in a given ecosystem should eventually converge into one large ring in order to gain maximum protection.}
	\item \textsl{If the Mullerian mimics do not converge into one large ring, then the presence of Batesian mimics could entice them to do so, by influencing the rings to converge.}
\end{enumerate}

Although there are many mathematical and stochastic models of mimicry in the biological literature, this model gives attention to the evolution of mimicry ring phenomenon from an artificial life perspective.

\subsection{FormAL Framework}
The \textsl{``FormAL framework"} is a collection of concepts taken from Peter Grogono's \gls{formal} project \citep{grogono2003} and are used to build a framework for this model. In FormAL, an \textbf{Agent} is a simulated organism. It is designed simply, but with capabilities of reproduction using genetic information and modification of genome between generations. There is also interaction between agents while being able to survive and reproduce in a challenging environment.

The framework consists of a three dimensional world where agents get complete freedom of movement defined from their genetic representation. This toroidal \textbf{space} is a 3D lattice of discrete points, divided in multiple cells, which can be visualized. A \textbf{cell} is a three dimensional cubical section of the hyperspace. The purpose of the cells is to avoid expensive distance calculations. As two agents are considered \textsl{``close"} to interact when they are in the same cell and \textsl{``distant"} otherwise. \textbf{Time}, being an integer \( (t \geq 0) \), advances in discrete steps in the simulation, where at each step the agents update themselves.

\subsubsection{Mobility}
An agent's position is calculated once during each step of update in time. The agents \(position\), \(force\), \(acceleration\) and \(velocity\) are all vector components. The \(force\) component is calculated from agent's mobility gene, based on which some agents are faster/slower than others, and it is used to compute agent's \(acceleration\). If the \(force\) and \(velocity\) are both zero, then the agent has no effect in motion. Otherwise, Newton's law is used to obtain the \(acceleration\), which is integrated to obtain the new \(velocity\) and new \(position\).

\subsection{The Prey: Models and Mimics}

For this simulation the preys are heliconius butterfly and the representation of their wing pattern is with the help of cellular automata (CA). Every prey organism contain a binary genetic representation of CA which generates a fully developed pattern of size 16 by 16 bits from its initial state. With this pattern the predator will identify the prey and store its level of palatability in memory. We choose CA as it can be easily represented with the help of a binary genome and evolutionary operations on the genomic representation, such as mutation and crossover can easily be applied. This 8 bit genome has a decimal range between 0 to 255. Each of this value is associated to a unique CA pattern. In generating the pattern of figure \ref{fig:cellular-automata-rule-30}, the genetic representation would be the `New state of center cell'. To store in Hopfield memory we take a linear representation of this 2-D pattern and to find similarity between two patterns we calculate hamming distance between their linear representations. 

\begin{figure}[h!]
	\centering
	\includegraphics{images/CARule30-large}
	\caption[Cellular Automata]{Cellular Automata Rule 30}
	\label{fig:cellular-automata-rule-30}
\end{figure}

\begin{table}[h!]
	\small
	\centering
	\setlength\tabcolsep{2pt}
	\begin{tabular}{| p{3cm} | c | c | c | c | c | c | c | c |}
	  \hline
	  Current Pattern & 111 & 110 & 101 & 100 & 011 & 010 & 001 & 000 \\ \hline
	  New state of center cell & 0 & 0 & 0 & 1 & 1 & 1 & 1 & 0 \\
	  \hline
	\end{tabular}
	\caption{Cellular Automata rule}
	\label{tab:cellular-automata-rule}
\end{table} 

\subsubsection{Species diversity}
\label{subsubsec:species-diversity}
Using CA based pattern representation, population of prey species with a specific pattern can be grouped as one single species. Also by restricting inter species reproduction we can control the diversity of patterns. But mutation is applied when similar species mate with each other, so new species born out of generations of existing species. That is why we have two separate mutation rate for reproduction of prey species. One being the ``Pattern Mutation Rate" (default values are mentioned in table \ref{tab:prey-control-parameters}) with which we control mutation of the first 8 bits of the genome while the ``Genome Mutation Rate" is used to control mutation of rest of the 9 bit genome. Similar efforts of multiple mutation rate at varying location has been used in developing Echo \citep{hraber1997}.

\subsubsection{Genome}
The Genome of prey species consists of 17 bits. The first eight bits represent the rule, which is used to generate CA pattern. Next two bits are used to represent palatability of the organism. Following six bits are the magnitude of force with which mobility of the organism is calculated. The 17th bit is used to evaluate reproduction capability of the organism.

\subsubsection{Reflection of punctuated equilibrium}
\label{subsubsec:reflection-of-punctuated-equilibrium}
Punctuated equilibrium is more inclined to cladogenesis instead of gradualism. Also Turner \citep{turner1988} emphasizes on punctuated equilibrium to describe the evolution of mimicry instead of phyletic gradualism. The design of the model under discussion also follows Turner's explanation in terms of evolving mimicry. As it can be observed, new CA patterns evolve from existing ones in prey population just by a single mutation in the pattern gene. Mimics do not follow a gradual process of evolution to look close to models but rather the change happens randomly through a single step mutation. The mutations that are favored, helps the mimics to survive while the unfavored ones fail to persist. It can be observed later in table \ref{tab:diff-in-pattern} how CA patterns of prey species can have vastly different configuration for a unit change in their representative gene, thus following the evolutionary process of punctuated equilibrium. 

\subsubsection{Palatability gene}
\label{subsubsec:genetic-palatability-representation}
The palatability of each prey species is fixed and has been represented with 2 bits (index 8 to 9) of the genome giving it a range of 0 to 3 with four levels of palatability. For the combinations of 00 and 01 palatability is true, while for 10 and 11 it is false.

\subsubsection{Interaction}
The prey have been defined to have many conglomerate behavior in the environment. Prey interaction with other prey species and with predators make the evolution of mimicry possible. Mobility of prey species and their reproduction capability are two important behaviors which result from interaction. 

\paragraph{Mobility}
The mobility genes of the prey consist of 6 bits. These six bits are used to calculate the force with which each prey try to move towards any neighborhood cell. The algorithm sorts all neighboring cell descending to the number of prey species. Then it selects the cell which contains the highest number of prey with zero predator. If all the neighboring cells contain predators, then the algorithm sorts the neighboring cells descending on the number of predators and chooses the one which contains the least. This implementation is to have a conglomerate behavior of all prey species, while running away from predators.

\paragraph{Reproduction}
Every prey species starts reproducing when it reaches the \textsl{``Reproductive age limit"}. If it is capable of reproducing, which is decided based on its 17th bit gene, the prey will randomly select another prey species with similar pattern and palatability from the same cell and mate with it, given the other prey is also capable of reproduction. A prey is created from the existing genome of the two prey by applying single point crossover operation. Mutation is performed separately on the pattern gene and the rest of the genome, with two different rates to control them using the values in table \ref{tab:prey-control-parameters}. So there is two point mutation for the genome. 

\begin{table}[h]
\small
\centering
\setlength\tabcolsep{2pt}
\begin{tabular}{| l | c |}
	\hline
		\textbf{Parameter} & \textbf{Value}\\ \hline
		Prey Size in the 3D FormAL environment & 2 to 5 \\ \hline
		Reproduction age limit & 100 \\ \hline
		Reproduction interval & 1000 \\ \hline
		Pattern Mutation Rate & 0.05 \\ \hline
		Genome Mutation Rate & 0.5 \\ \hline
		Demise Age & 2000 \\
	\hline
\end{tabular}
\caption{Parameters to control prey population and visibility.}
\label{tab:prey-control-parameters}
\end{table}

\subsection{Predator}

Predators in the system are designed to provide selection pressure to \textit{models} and \textit{mimics} for the evolution of mimicry. Similar to prey species, they are agents in the FormAL environment capable of mobility and reproduction. In addition, these agents are equipped with Hopfield Network Memory to be able to learn and recognize patterns of the prey species. Their mobility and reproduction capability are controlled at the genetic level, while their memory is not genetically controlled, as we could not find a suitable encoding for the genetic representation of Hopfield Network. Every new predator is born with zero memory and with no inheritance from parents. A set of parameters are defined to control predators' population and learning ability in the environment (table \ref{tab:predator-control-parameters}).

\subsubsection{Learning}
The objective of a predator's interaction with prey is always to consume it. But based on the prey's pattern and palatability, the predator will either be able to consume it or throw it back to the environment. At this event the predator needs to learn the pattern with which the prey has been represented. The pattern represents palatability of the prey species, at least to the predator. Every time a new interaction is made by the predator its memory is initialized with all the existing pattern that has already been encountered and the new one. The learning procedure used for this memory is Hebbian Learning \citep{hebb1949}, which represents a purely feed-forward, unsupervised learning. Initially the weights of the Hopfield Network are all set to zero. Using Hebbian rule, the outer product of the input - output vector pairs are calculated for each pattern. The outer vector matrix of all the patterns are summed to come up with the final weight matrix.

%This section needs to use similar variables as mentioned in the above section on Hopfield Network.
\paragraph{Input to memory}
Each prey contains an evolving cellular automata which is represented by a binary genome. This two dimensional pattern is serialized to be available as a one dimensional binary array, which is taken as input for any predator organism trying to interact with the prey. This binary representation of the pattern gets converted to a bipolar representation. Each input pattern consists of \(\textit{m} \times \textit{n} = \textit{mn}\) components, each component representing one pixel of the pattern (\textit{m} and \textit{n} representing each dimension). The \textit{m} by \textit{n} pattern configuration is serialized by putting all row vectors in one single row sequentially.

\paragraph{Predator attack algorithm}
As soon as a predator reaches its attack age it selects random prey species around its vicinity and starts attacking them. This attack process also involves recognition of prey pattern. Two parameters have been defined to limit predator memorization and recognition process as both of these processes are computationally expensive. The ``Hopfield Minimum Memory Size" is the number of memory a predator needs to store before making intelligent decisions about attacking a new prey species. When a predator is born, it starts attacking prey without any caution. But after every attack the predator will store its pattern and palatability level inside its memory. As soon as it reaches the minimum memory size, it will start making intelligent decision about attacking the next prey species. It will try to recognize the pattern and if found palatable, the prey will be consumed. Otherwise prey will be thrown back into the environment. If the pattern is not recognized predator will try to consume it and in the process will store its palatability and pattern into memory. In this way the predator memory is limited to \textsl{``Hopfield Maximum Memory Size"}. After reaching this memory predator will not store any more new pattern but will try to associate with the existing ones it has already stored.

\paragraph{Genome}
Each predator has a 5 bit genome. The first 4 bits are for mobility, while the last bit controls reproduction capability of each species.

\paragraph{Mobility}
Movement behavior of a predator is calculated from its genome. The first 4 bit converted to decimal is the magnitude of force (varying from 0-15) with which it will move towards the maximum crowd of prey present within its neighborhood. Number of bits are less than prey (6 bits) to reduce their maximum speed. If no prey is present in the neighborhood then this force is active in trying to keep predators distributed all over the cells with a constant mobile behavior. This behavior of predator has been designed to increase predator prey interaction in the simulation in terms of one agent chasing the other for survival of species. 

\paragraph{Reproduction}
The  fifth gene of the predator is used to represent their capability of reproduction.  Depending on its binary value a predator in the simulation will or will not be able to reproduce. The reproduction process for predators is similar to prey species as the learning capability of predators do not have any genetic representation. Similar to the prey the predator also have the \textsl{``Reproduction Interval"} and the \textsl{``Reproduction Age Limit"} which is the minimum age it has to reach before starting to mate. Using these parameters we can control the population of predators and by which we also control the rate of predation on prey species.

\begin{table}[h]
\small
\centering
\begin{tabular}{| l | c |}
	\hline
		\textbf{Parameter} & \textbf{Value} \\ \hline
		Minimum Memory Size & 2 to 6 \\ \hline
		Maximum Memory Size & 10 \\ \hline 
		Hopfield Maximum Iterations & 20 \\ \hline
		Attack Age & 500 \\ \hline
		Attack Interval & 100 \\ \hline
		Genome Mutation rate & 0.3 \\ \hline
		Reproduction Age Limit & 500 \\ \hline
		Reproduction Interval & 1000 to 3000 \\ \hline
		Demise Age & 2000 to 7000 \\
	\hline
\end{tabular}
\caption{Parameters to control predator population and pattern recognition capability.}
\label{tab:predator-control-parameters}
\end{table}

This model has been designed to come up with efficient results and achieve the main objective, \textit{evolution of mimicry}. Creation and transformation of different mimicry ring and also the dynamics of it has been integrated to achieve interesting results. This model can also be considered as a complex adaptive system similar to Holland's work on Echo \citep{holland1996}. The seven basics of a complex adaptive system which are: Aggregation, Tagging, Nonlinearity, Flow, Diversity, Internal Models and Building blocks \citep{holland1996} are present in this model. Individual components of this model such as the different types of agents and their properties can be considered as \textit{building blocks}. Each prey species are \textit{tagged} with individual pattern and palatability with which predators recognize them. We are providing different properties, behaviors and goals to the agents but setting them free in the environment to observe their \textit{aggregate} behavior, resulting in \textit{non-linear} or unpredictable outcome. The model has its \textit{flow} as it progresses in time. Also there is \textit{diversity} of prey species in the environment.

\begin{figure*}[t!]
	\centering
	\includegraphics[resolution=300]{images/simTime10k-2Prey}
	\caption[Population distribution of mimicry rings (2 prey species, 10k iterations)]{Population distribution of mimicry rings, initialized with 2 prey species, 10k iterations}
	\label{fig:plot-2-prey}
\end{figure*}

\section{The Results}
\label{section:results}

Data and analysis in this simulation has been concentrated on evaluating whether evolution of mimicry has taken place. This evaluation can be made with the number of different rings that has been created and the size of each of those rings along with the population of palatable and unpalatable species. Also it can be established whether Batesian Mimicry and Mullerian Mimicry have taken effect by analyzing the data set of these populations.

\subsection{Mimicry Ring Reports}
The mimicry ring reports consist entirely of the population of prey species categorized according to pattern and palatability. Data is stored at time interval of 10 iterations. As the number of rings that get generated reaches as many as 50 or more, and all the population of ring do not last for the entire simulation, so while storing data we have taken the most populous of the surviving 8 rings to plot. Mimicry Ring Hamming distance between patterns is 10 \% of the pattern size.

\subsection{Initial configuration with two prey species}
\label{subsec:init-conf-2prey}
% Put the table here
\begin{table}[h]
\small
\centering
\setlength\tabcolsep{2pt}
\begin{tabular}{| p{1.75cm} | m{1.75cm} | p{1cm} | p{.5cm} | p{1.5cm} | p{.5cm} |}
  \hline
   														&\multicolumn{3}{c|}{Prey configuration} 																	
   														& \multicolumn{2}{c|}{\parbox[c][2.4em][c]{2cm}{Predator \\ configuration}} \\ \hline
  \multirow{2}{*}{Population} & Rule110 (Palatable) & \parbox[c]{2.1em}{\includegraphics{images/CARule110}} & 108 
  														& \multicolumn{2}{c|}{\multirow{2}{*}{10}} \\ \cline{2-4}
  					 									& {\parbox{1.75cm}{Rule30\\ (Unpalatable)}}& \parbox[c][2.4em][c]{2.1em}{\includegraphics{images/CARule30}}  & 108 
  					 									& \multicolumn{2}{c|}{}\\ \hline
  \multirow{2}{*}{Reproduction} & Age Limit & \multicolumn{2}{c|}{100}  & \multicolumn{2}{c|}{500} \\ \cline{2-6}
  						 									& Interval  & \multicolumn{2}{c|}{1000} & \multicolumn{2}{c|}{1200} \\ \hline
  \multirow{2}{*}{\parbox{1.75cm}{Mutation\\ Rate}} & Pattern   & \multicolumn{2}{c|}{0.05} & \multicolumn{2}{c|}{\multirow{2}{*}{0.3}} \\ \cline{2-4}
  						 									 & Genome    & \multicolumn{2}{c|}{0.5}  & \multicolumn{2}{c|}{} \\ \hline
  Demise Age	 									 & \multicolumn{3}{c|}{2000}							& \multicolumn{2}{c|}{2500} \\ \hline
  {\parbox[c][2.3em][c]{1.75cm}{Minimum\\ Attack Age}}	& \multicolumn{3}{c|}{} 						    & \multicolumn{2}{c|}{500} \\ \hline
  \multirow{2}{*}{\parbox{1.75cm}{Memory\\ Configuration}} & \multicolumn{3}{c|}{} 					& Minimum & 2 \\ \cline{5-6}
   																			& \multicolumn{3}{c|}{} 					& Maximum & 10 \\ \hline  
\end{tabular}
\caption{Agent configuration of 2 prey species}
\label{tab:config-table-2-prey}
\end{table}

The set of parameters in table \ref{tab:config-table-2-prey} were carefully selected to be the initial condition for this run of the simulation. This test has been done with two sets of prey species with very different Cellular Automata pattern and with opposite palatability and equal population. To control reproduction of the prey species their age limit has been set to 100 iterations into the time the species were alive. And the reproduction interval was set to 1000 iterations.

Pattern mutation rate has been set to a minimal level of 0.05 as by increasing this variable it is possible to increase the size of the number of mimicry rings present in the simulation. The genome mutation rate controls the rate at which genome of the child prey species will deviate from their parents.

Prey demise age has been kept to 2000 iterations while predator demise age is set to 2500. Predators in this simulation generate selection pressure for the evolution of mimicry. So the longer a predator is present in the simulation it will be making intelligent decisions. Using this rate of demise for predator we were able to create successful mimetic population of prey species.

Initial population of predator species has been set to 10 which is in accordance with the prey population in the simulation. The reason for such low number of predator is, unlike prey species which are consumed by predators, there is no cause for the predator species to die except their natural cause of death, that is to reach their demise age. So predator population can explode very easily. That is why their population is controlled in a restrictive manner with the help of high reproduction age limit and reproduction age interval.


\begin{figure*}[t!]
	\centering
	\includegraphics[resolution=300]{images/simTime8k-4Prey-unp}
	\caption[Population distribution of mimicry rings(4 prey species all unpalatable)]{Population distribution of mimicry rings, initialized with 4 prey species all unpalatable.}
	\label{fig:plot-4-prey-unp}
\end{figure*}

The plot in Figure \ref{fig:plot-2-prey} is simulation time verses prey population after running it for 10000 iterations. With the initial configuration in the above table we can observe that multiple rings of prey population have been created. Two prey species are considered to be in a ring if their CA pattern have Hamming distance within 10 bits. Population of palatable species have been represented with line curve while population of unpalatable species have been presented with dotted curve. Different signs of squares, triangles and diamonds have been used to distinguish between species of prey population. The simulation was initiated with two prey species having CA rule of 110 and 30, being palatable and unpalatable respectively. Rule 110 and 30 has been used as their phenotype is distinctly different from each other and Hopfield Network will easily distinguish them. Over time the population of CA Rule 30 dominates the population (Figure \ref{fig:plot-2-prey}) as most predators recognize it as unpalatable. Similarly a palatable population of CA Rule 30 or within the same ring of palatable species starts rising, while at one point overlaps the population of CA Rule 110 (Time: 4000 approx.). CA Rule 110 was initialized as a set of palatable species.

We can observe from the above result that the evolution of mimicry has taken effect. A population of mimics were successfully able to exceed the population of other prey species, and the reason being, avoidance by predators of prey pattern similar to unpalatable ones. We can conclude that Batesian mimicry has taken effect in the simulation.

The number of rings in this simulation makes a slow increase from 2 at the initial configuration to 27 rings at the end of 10000 iterations. A small change in CA genetic representation can have a very large effect in terms of the phenotype of the pattern with which the prey is represented. For example if we take a look at the set of almost similar pattern genotype with vastly different phenotype in table \ref{tab:diff-in-pattern}.

%CARule table
\begin{table}[h]
\small
\centering
\setlength\tabcolsep{2pt}
\begin{tabular}{|l|c|c|c|}
  \hline
  CA Rule & \(60 \equiv 00111100\) & \(61 \equiv 00111101\) & \(62 \equiv 00111110 \) \\ \hline
  Pattern & \parbox[c][2em][c]{2.3em}{\includegraphics{images/CARule60}} 
  				& \parbox[c][2em][c]{2.3em}{\includegraphics{images/CARule61}} 
  				& \parbox[c][2em][c]{2.3em}{\includegraphics{images/CARule62}}\\
  \hline
\end{tabular}
\caption{Difference in prey pattern genotype and phenotype}
\label{tab:diff-in-pattern}
\end{table}

All the patterns in table \ref{tab:diff-in-pattern} have a genetic bit difference of 1. So by a single mutation there can be three different set of phenotype for a child organism from its parent. This is largely the reason for the increased number of mimicry rings created in the simulation. Only the 8 most populous rings are presented in the graphs with population verses simulation time.

To evaluate the simulation at a more complex level we increased the prey population to 900, consisting of 6 different species with very different pattern configuration. To boost predator-prey interaction we also increased the number of predator population to 30. This resulted in an enormous diversity of species where the total number of mimicry rings reached nearly 50. Details of this result can be found in \citep{mohiulThesis2011}.

\subsection{Initial configuration with only unpalatable species}
\label{subsec:init-conf-only-unp}

To further observe the effects of mimicry ring we initialize the simulation with all four unpalatable prey species. As explained earlier the minimum memory configuration is also set to four in accordance to the initial number of prey species. Rest of the parameters remain quite unchanged.

The results according to figure \ref{fig:plot-4-prey-unp} are much expected. The population of unpalatable species have prevailed. After nearly 8000 iterations we can see unpalatable species of CA rule 55, 110, 30 and 190 have prevailed. All of their palatable counter parts are also increasing their population deceiving the predators. 

This experiment is an ideal scenario for observing Mullerian mimicry. Mullerian mimicry occurs between multiple species of unpalatable prey population. From Franks and Noble \citep{franks2003}, we note that multiple Mullerian mimicry rings are expected to converge into one large ring through the evolutionary process of punctuated equilibrium. But in this experiment as the predator's `Minimum Memory Configuration' is set to four, all predators have the capability to recognize four prey patterns before starting to make intelligent decision of consuming them. By setting `Minimum Memory Configuration' to one, also increasing `Predator Demise Age' to 7000 and decreasing predator's `Reproduction Age Interval' to 1500, we run the simulation for 6000 iterations, and there was no sign for all prey population to converge into one large ring. All four unpalatable prey population have a very dominant presence in the simulation. Even after reducing predator minimum memory to one pattern, different population of predators become familiar with different prey patterns, which resulted in the existence of multiple Mullerian mimicry ring instead of a single one.

In contrast when the simulation was initiated with only palatable species all population of prey were consumed by predators at nearly 7000 iterations, details of which can be found in \citep{mohiulThesis2011}.

\subsection{Analysis}
For all possible initial conditions, Batesian mimicry has taken effect. It can be observed that for every ring of unpalatable species there is an existence of the palatable ring racing to reach the population count of its unpalatable counterpart. Effects of Mullerian mimicry can also be observed best for the experiment initialized with only unpalatable prey species. We initialized the model with 4 rings of unpalatable species with no palatable ones and after nearly 10K iterations, all of the initial unpalatable rings have survived with dominance. The cause of this behavior can be explained by the minimum number of patterns that each predator can store in memory, which was set to four. So this parameter was reduced to one to observe whether it is possible to converge all different unpalatable rings into one large ring, when predators are capable of memorizing only a single pattern. But as it turned out, the phenomena of ``a single large ring" does not occur because different predators recognize different patterns resulting in multiple divergent Mullerian mimicry rings. It can be concluded that our results are consistent with those of Franks and Noble \citep{franks2003}, that multiple Mullerian mimics do not converge into one large ring. These claims can only be made within the limits of this simulation.

\section{Conclusion}
\label{section:conclusion}
Analysis of the results tell us that we have successfully been able to simulate the evolution of mimicry. In addition to that, this model provides a more accurate simulation of the fascinating natural process of mimicry rings and their shift in population. This model also verifies the theory of Turner in explaining the evolution of mimicry with punctuated equilibrium \citep{turner1988}.

\bibliographystyle{apalike}
\bibliography{references}

\printglossaries
\phantomsection \label{acronyms}

\end{document}
