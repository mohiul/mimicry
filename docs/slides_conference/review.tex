\section{Review}

\frame
{
	\begin{center}
		\LARGE A Review
	\end{center}
}

\frame
{
	\frametitle{Review}
	
	\begin{itemize}
		\item Evolutionary Computation
			\begin{itemize}
				\item Turing (1948):
					\begin{itemize}
						\item ``Intelligent Machinery", Expression ``genetical or evolutionary search"
					\end{itemize}
				\item Bremermann (1962): Solve optimization problems.
				\item Rechenberg (1964): Evolutionary strategies
				\item L. Fogel (1965): Evolutionary Programming
				\item Holland (1975): Genetic Algorithms
				\item Koza (1992): Genetic Programming				
			\end{itemize}
		\item Artificial Life
			\begin{itemize}
				\item Introduced by Christopher G. Langton (1987)
				\item Complex Adaptive System
				\item Echo
			\end{itemize}
	\end{itemize}
}

%\frame
%{
%	\frametitle{Evolutionary Computation}
%	\framesubtitle{History}
%	
%	\begin{itemize}
%		\item Turing (1948):
%			\begin{itemize}
%				\item ``Intelligent Machinery"
%				\item Expression ``genetical or evolutionary search"
%			\end{itemize}
%		\item Bremermann (1962):
%			\begin{itemize}
%				\item Solve optimization problems.
%				\item Minimize a real-valued function.
%			\end{itemize}
%	\end{itemize}	
%}
%
%\frame
%{
%	\frametitle{Evolutionary Computation}
%	\framesubtitle{History}
%
%	\begin{itemize}
%		\item Rechenberg (1964): Evolutionary strategies
%			\begin{itemize}
%				\item Self adaptation of strategy parameters.
%				\item Aerodynamic wing design.
%			\end{itemize}
%		\item L. Fogel, Owens and Walsh (1965): Evolutionary Programming
%			\begin{itemize}
%				\item Generate Artificial Intelligence by simulating evolution as a learning process.
%				\item Off springs are created via mutation.
%				\item No recombination operator.
%			\end{itemize}
%	\end{itemize}
%}
%
%\frame
%{
%	\frametitle{Evolutionary Computation}
%	\framesubtitle{History}
%
%	\begin{itemize}
%		\item Holland (1975): Genetic Algorithms
%			\begin{itemize}
%				\item Schema theorem.
%				\item Wide set of applications.
%				\item Holland's ``Adaptation in natural and artificial systems"
%				\item Effective to solve optimization problems.
%				\item Ineffective: solution set does not have gnomic representation.
%			\end{itemize}
%		\item Koza (1992): Genetic Programming
%			\begin{itemize}
%				\item Automated method for creating computer program.
%				\item Applied to machine learning: prediction and classification.
%				\item Very slow process.
%				\item Mutation is mostly avoided.
%				\item Recombination operator work by exchanging trees.
%			\end{itemize}
%	\end{itemize}
%}
%
\frame
{
	\frametitle{Evolutionary Algorithm}

	\begin{itemize}
		\item Strategy: survival of the fittest.
		\item Fitness function
		\item Variation operators:
			\begin{itemize}
				\item Mutation
				\item Recombination
			\end{itemize}
		\item Fundamental forces:
			\begin{itemize}
					\item ``Variation operators create diversity; facilitates novelty."
					\item ``Selection acts as a force pushing quality."
			\end{itemize}
	\end{itemize}
}

%\frame
%{
%	\frametitle{Artificial Life}
%	\framesubtitle{}
%
%	\begin{itemize}
%		\item Introduced by Christopher G. Langton.
%		\item At `International Conference on the Synthesis and Simulation of Living Systems', Los Alamos National laboratory 1987.
%		\item Described by Taylor as a tool for biological inquiry.
%	\end{itemize}
%}
%
%\frame
%{
%	\frametitle{Artificial Life}
%	\framesubtitle{}
%
%	\begin{itemize}
%		\item Wetware system (Molecular level):
%			\begin{itemize}
%				\item Natural life or derived from it.
%				\item Artificial evolutionary process toward the production of RNA molecules.
%			\end{itemize}
%		\item Software system (Cellular level):
%			\begin{itemize}
%				\item Initial contribution: John von Neumann.
%				\item Self reproducing, computation universal Cellular Automata.
%			\end{itemize}
%		\item Hardware system (Organism level):
%			\begin{itemize}
%				\item Organism's sensory and nervous system: body and environment.
%				\item Geometric, mechanical, dynamical, thermal properties.
%			\end{itemize}
%	\end{itemize}
%}

\frame
{
	\frametitle{Complex Adaptive System}
	\framesubtitle{}

	\begin{itemize}
		\item Special cases of complex systems.
		\item Diverse, multiple interconnected elements.
		\item Adaptive: capacity to change and learn from experience.
		\item Examples:
		\begin{itemize}
			\item Stock market.
			\item Social insect and ant colonies.
			\item Biosphere and the ecosystem.
			\item Brain and the immune system.
			\item Cell and the developing embryo.
			\item Manufacturing businesses.
			\item Human social group-based endeavor in a cultural and social system.
			\item Large-scale online systems, collaborative tagging or social bookmarking systems.
		\end{itemize}
	\end{itemize}
}
%
%\frame
%{
%	\frametitle{Complex Adaptive Systems}
%	\framesubtitle{Criteria}
%
%	\begin{itemize}
%		\item ``Large number of parts undergoing a kaleidoscopic array of simultaneous non linear interactions."
%		\item ``Impact in human affairs: aggregate behavior; behavior of the whole."
%		\item ``Interactions evolve over time: parts adapt to survive in the environment"
%		\item ``Complex adaptive systems anticipate."
%	\end{itemize}
%}

\frame
{
	\frametitle{Echo}
	\framesubtitle{Criteria}

	\begin{itemize}
		\item Simplicity, thought experiment, primitive internal model.
		\item Fitness: an evolving criteria.
		\item Flexible architecture: incorporate well studied mathematical models.
		\begin{itemize}
			\item Dawkins' biological arms race.
			\item Brower's survival of mimics.
			\item Wicksell's Triangle and overlapping generation models in economics.
			\item Prisoner's dilemma game in political science.
			\item Holland's two-armed bandits in operational research.
			\item Perelson's antigen-body matching in immunology.
		\end{itemize}
		\item Amenable to mathematical analysis.
	\end{itemize}
}