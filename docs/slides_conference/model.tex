\section{The Model}

\frame
{
	\begin{center}
		\LARGE The Model: Evolution of Mimicry
	\end{center}
}

\frame
{
	\frametitle{The Model}
	\framesubtitle{Evolution of Mimicry}
	
	\begin{itemize}
		\item \textbf{Objective:} Build an \textit{agent based} Artificial Life model for simulating the evolution of mimicry.	
		\item Two species of agents:
			\begin{enumerate}
				\item Prey
					\begin{itemize}
						\item Model
						\item Mimic
					\end{itemize}
				\item Predator
			\end{enumerate}
		\item Prey pattern representation: Cellular Automata.
		\item Predator pattern recognition: Hopfield Network.
		\item Environment: Visual representation, 3D, toroidal.
	\end{itemize}	
}

\subsection{FormAL}

\frame
{
	\frametitle{FormAL Framework}
	
	\begin{itemize}
		\item Ideas from Peter Grogono's Formal Artificial Life (FormAL) project.
		\item \textbf{Goal:}		
			\begin{itemize}
				\item Study the emergence of complexity.
				\item No variable unless genetically controlled or influenced. \textit{Principal not followed for Hopfield Network.}
			\end{itemize}
		\item \textbf{Agents:}
			\begin{itemize}
				\item Simulated organism.
				\item Reproduce itself using genetic information.
				\item Capable of modifying structures of genome between generations.
				\item Interaction with other agents.
				\item Survive and reproduce in a challenging environment.
			\end{itemize}
	\end{itemize}
	
}

\frame
{
	\frametitle{FormAL Framework}
	\framesubtitle{Environment - Visual representation - Front}

	\begin{figure}[H]
		\centering
		\includegraphics[scale=0.40]{../tex/images/cells-front}
		\caption{Three dimensional representation of the environment divided in cells. Presence of different species of agents inside.}
	\label{tab:3-d-environment-images-1}
	\end{figure}	
}

\subsection{Prey}

\frame
{
	\frametitle{The Prey}
	\framesubtitle{Mimics and Models}
	
	\begin{itemize}
		\item Agent in the FormAL environment.
		\item Genetic representation of pattern with Cellular Automata.
		\item Creates diversity of species.
		\item Pattern evolution is in the process of punctuated equilibrium.
		\item Mobility and reproduction capability controlled genetically.
	\end{itemize}
}

\frame
{
	\frametitle{The Prey: Mimics and Models}
	\framesubtitle{Pattern Representation}

	\begin{figure}[H]
		\centering
		\includegraphics[scale=3]{../tex/images/CA_rule30s}
		\caption{Cellular Automata Rule 30.
		Image source: \href{http://en.wikipedia.org/wiki/Cellular_automata}{Wikipedia}}
		\label{fig:cellular-automata-rule-30}
	\end{figure}
	
	\begin{table}
		\centering
		\begin{scriptsize}
		\begin{tabular}{| p{2cm} | c | c | c | c | c | c | c | c |}
		  \hline
		  Current Pattern & 111 & 110 & 101 & 100 & 011 & 010 & 001 & 000 \\ \hline
		  New state of center cell & 0 & 0 & 0 & 1 & 1 & 1 & 1 & 0 \\
		  \hline
		\end{tabular}
		\end{scriptsize}		
		\caption{Cellular Automata rule}
		\label{tab:cellular-automata-rule}
	\end{table}
}

\frame
{
	\frametitle{The Prey: Mimics and Models}
	\framesubtitle{Species Diversity}

	\begin{itemize}
		\item Pattern genome is 8 bit binary. Decimal range 0 to 255.
		\item 256 unique CA pattern.
		\item Linear representation of pattern stored in Hopfield Network.
		\item Pattern similarity: Hamming distance between linear representation.
		\item Single species: group of prey with a specific pattern.
		\item Inter species reproduction: restricted to control diversity of patterns.
		\item ``Pattern Mutation Rate": control diversity of new species.
	\end{itemize}
}

\frame
{
	\frametitle{Prey Pattern}
	\framesubtitle{Genotype vs. Phenotype}
	
	\begin{itemize}
		\item Genetic bit difference of one.
		\item Vastly different phenotype.
	\end{itemize}
	
	\begin{table}
	\centering
	\begin{scriptsize}
	\begin{tabular}{|l|c|c|c|}
	  \hline
	  CA Rule & \(60 \equiv 00111100\) & \(61 \equiv 00111101\) & \(62 \equiv 00111110 \) \\ \hline
	  Pattern & \parbox[c]{2.1em}{\includegraphics[scale=0.40]{../tex/images/CARule60}} 
	  				& \parbox[c]{2.1em}{\includegraphics[scale=0.40]{../tex/images/CARule61}} 
	  				& \parbox[c]{2.1em}{\includegraphics[scale=0.40]{../tex/images/CARule62}}\\
	  \hline
	\end{tabular}
	\end{scriptsize}
	\caption{Difference in prey pattern genotype and phenotype}
	\label{tab:diff-in-pattern}
	\end{table}
}

\frame
{
	\frametitle{The Prey: Mimics and Models}
	\framesubtitle{Genome}
	
	\begin{itemize}
		\item 17 bit prey Genome
	\end{itemize}
	
	\begin{table}[H]
	\centering
	\begin{scriptsize}
	\begin{spacing}{1.5}
	\begin{tabular}{|c|c|c|c|}
		\hline
			\textbf{Pattern(8)} & \textbf{Palatability(2)} & \textbf{Mobility(6)} & \textbf{Reproduction(1)} \\ \hline
			10101101					 	& 							01		 		 & 			110001					&					1						 		 \\ \hline
	\end{tabular}
	\end{spacing}
	\end{scriptsize}
	\caption{Distribution and purpose of each gene of the 17 bit prey genome.}
	\label{tab:prey-genome}
	\end{table}
}

\frame
{
	\frametitle{The Prey: Mimics and Models}
	\framesubtitle{Punctuated Equilibrium}
	
	\begin{itemize}
		\item Punctuated Equilibrium:
			\begin{itemize}
				\item inclined to cladogenesis instead of gradualism.
				\item Turner's emphasis on punctuated equilibrium to explain evolution of mimicry.
				\item CA pattern evolution: single mutation in the pattern genome.
				\item Change of pattern: 
					\begin{itemize}
						\item not gradual
						\item arbitrary discontinuous
					\end{itemize}
			\end{itemize}
	\end{itemize}
}

\subsection{Predator}

\frame
{
	\frametitle{The Predator}
	
	\begin{itemize}
		\item Agent in the FormAL environment.
		\item Provide selection pressure for the evolution of mimicry.
		\item Equipped with Hopfield Network Memory.
		\item Mobility and reproduction capability controlled genetically.
		\item Unable to represent pattern recognition capability with genome.
		\item New predators are born with zero memory, as memory is not inherited.
	\end{itemize}
}

\frame
{
	\frametitle{The Predator}
	\framesubtitle{Learning}

	\begin{itemize}
		\item Predator's interaction objective with prey is consumption.
		\item Consumption is based on palatability.
		\item If unable to consume prey is thrown back into environment.
		\item Store prey pattern into memory with the associated palatability.
		\item New pattern learned with Hebbian Learning.
	\end{itemize}
}

\frame
{
	\frametitle{The Predator}
	\framesubtitle{Design of Memory}
	
	\begin{itemize}
		\item Input to Memory:
			\begin{itemize}
				\item Each prey has an evolving CA represented by a binary Genome.
				\item 2D pattern is serialized to a 1D binary array.
				\item Binary representation converted to bipolar representation.
			\end{itemize}
		\item Pattern Recognition with Hopfield Network:
			\begin{itemize}
				\item Learning: Apply Hebbian learning to calculate weights.
				\item Initialization: Input to network initialized with input vector.
				\item Iterate Until Convergence: Asynchronous update of each neuron. Input: previous state.
				\item Output: Finally a pattern is set as output when the network reaches convergence.
			\end{itemize}
	\end{itemize}
}

\frame
{
	\frametitle{The Predator}
	\framesubtitle{Attack Algorithm}
	
	\begin{itemize}
		\item Agent reaches `Minimum Attack Age' it starts hunting.
		\item Select a random prey within vicinity (same cell).
		\item Involves recognition of prey pattern.
		\item Pattern memorization and recognition process: computationally expensive.
		\item Two parameters to limit 
			\begin{itemize}
				\item Hopfield Minimum Memory Size (value 2 to 6)
				\item Hopfield Maximum Memory Size (value 10)
			\end{itemize}
		\item New predator attacks without caution. 
		\item Attacks everyone and in the process store pattern and palatability.
		\item When memory reaches `Hopfield Minimum Memory Size': intelligent selection.
	\end{itemize}
}
