%% Based on a TeXnicCenter-Template by Gyorgy SZEIDL.
%%%%%%%%%%%%%%%%%%%%%%%%%%%%%%%%%%%%%%%%%%%%%%%%%%%%%%%%%%%%%

%----------------------------------------------------------
%
\documentclass[letterpaper,landscape,titlepage,leqno]{slides}%
%
%----------------------------------------------------------
% This is a sample document for the LaTeX Slides Class
% Class options
%       --  Body text point size (normalsize) is 27 (default)
%           and can not be adjusted to any other value.
%       --  Paper size:  letterpaper (8.5x11 inch, default)
%                        a4paper, a5paper, b5paper,
%                        legalpaper, executivepaper
%       --  Orientation (portrait is the default):
%                        landscape
%       --  Quality:     final(default), draft
%       --  Title page:  titlepage, notitlepage
%       --  Columns:     onecolumn (default), [twocolumn is not avalible]
%       --  Equation numbering (equation numbers on the right is the default)
%                        leqno (equation numbers on the left)
%       --  Displayed equations (centered is the default)
%                    fleqn (flush left)
%
%  \documentclass[a4paper,fleqn]{slides}
%
%  The slides are separated from each other by the slide
%  environment, see below:
%
\usepackage{amsmath}%
\usepackage{amsfonts}%
\usepackage{amssymb}%
\usepackage{graphicx}
%------------------------------------------------------------------
\hfuzz5pt % Don't bother to report overfull boxes < 5pt
\newtheorem{theorem}{Theorem}
\newtheorem{corollary}[theorem]{Corollary}
\newtheorem{definition}[theorem]{Definition}
\newtheorem{example}[theorem]{Example}
\newtheorem{exercise}[theorem]{Exercise}
\newtheorem{lemma}[theorem]{Lemma}
\newtheorem{proposition}[theorem]{Proposition}
\pagestyle{plain}
%%% --------------------------------------------------------------
\begin{document}
\title{Modeling the Evolution of Mimicry}
\author{Mohiul Islam}
\date{\today}
\maketitle
\section{The Inspiration}

\frame
{
	\frametitle{Mimicry}
}

\subsection{Batesian Mimicry}

\frame
{
	\frametitle{Batesian Mimicry}
}

\subsection{Mullerian Mimicry}

\frame
{
	\frametitle{Mullerian Mimicry}
}

\subsection{Evolutionary Dynamics}

\frame
{
	\frametitle{Evolutionary Dynamics}
}

\frame
{
	\frametitle{Mimicry Rings}
}
%Chapter describing the model
\chapter{Modeling the evolution of mimicry}

\section{Results}

\frame
{
	\begin{center}
		\LARGE The Results
	\end{center}
}

\frame
{
	\frametitle{Results}

	\begin{itemize}
	  \item \textbf{Objective}: Evaluate evolution of mimicry.
	  \item \textbf{Evaluation process}:
			\begin{itemize}
			  \item Calculate number of mimicry rings.
			  \item Calculate size of mimicry rings:
					\begin{itemize}
					  \item Population of \textit{palatable} species.
					  \item Population of \textit{unpalatable} species.
					\end{itemize}			  
			\end{itemize}
		\item \textbf{Report parameters}:
			\begin{table}
			\centering
			\begin{scriptsize}
			\begin{spacing}{1.5}
			\begin{tabular}{| l | c |}
				\hline
					\textbf{Parameter} & \textbf{Value} \\ \hline
					Mimicry Ring hamming distance & 10 \% of the Pattern Size \\ \hline
					Number of Rings to report & 8 \\
				\hline
			\end{tabular}
			\end{spacing}
			\end{scriptsize}
			\caption{Parameters to mimicry ring report.}
			\label{tab:ring-report-control-parameters}
			\end{table}		
	\end{itemize}
}

\subsection{Experiments}

%-----------------------------
%----- Two Prey Species ------
%-----------------------------

\frame
{
	\frametitle{Two Prey Species}
	\framesubtitle{Initial Configuration}

	\begin{table}
	\centering
	\begin{tiny}
	\begin{spacing}{1.5}
	\begin{tabular}{|l|l|c|c|l|c|}
	  \hline
	   														&\multicolumn{3}{c|}{Prey configuration} 																	
	   														& \multicolumn{2}{c|}{Predator configuration} \\ \hline
	  \multirow{2}{*}{Population} & Rule110 (Palatable) & \parbox[c]{2.1em}{\includegraphics[scale=0.30]{../tex/images/CARule110}} & 108 
	  														& \multicolumn{2}{c|}{\multirow{2}{*}{10}} \\ \cline{2-4}
	  					 									& Rule30 (Unpalatable)& \parbox[c]{2.1em}{\includegraphics[scale=0.30]{../tex/images/CARule30}}  & 108 
	  					 									& \multicolumn{2}{c|}{}\\ \hline
	  \multirow{2}{*}{Reproduction} & Age Limit & \multicolumn{2}{c|}{100}  & \multicolumn{2}{c|}{500} \\ \cline{2-6}
	  						 									& Interval  & \multicolumn{2}{c|}{1000} & \multicolumn{2}{c|}{1200} \\ \hline
	  \multirow{2}{*}{Mutation Rate} & Pattern   & \multicolumn{2}{c|}{0.05}& \multicolumn{2}{c|}{\multirow{2}{*}{0.3}} \\ \cline{2-4}
	  						 									 & Genome    & \multicolumn{2}{c|}{0.5} & \multicolumn{2}{c|}{} \\ \hline
	  Demise Age	 									 & \multicolumn{3}{c|}{2000}						& \multicolumn{2}{c|}{2500} \\ \hline
	  Minimum Attack Age						 & \multicolumn{3}{c|}{} 						    & \multicolumn{2}{c|}{500} \\ \hline
	  \multirow{2}{*}{Memory Configuration} & \multicolumn{3}{c|}{} 					& Minimum & 2 \\ \cline{5-6}
	   																			& \multicolumn{3}{c|}{} 					& Maximum & 10 \\ \hline  
	\end{tabular}
	\end{spacing}
	\end{tiny}
	\caption{Agent configuration of 2 prey species}
	\label{tab:config-table-2-prey}
	\end{table}
}

\frame
{
	\frametitle{Two Prey Species}
	\framesubtitle{Population vs. Time (10k)}
	
	\begin{figure}
		\centering
		\includegraphics[scale=0.25]{../tex/images/simTime10k-2Prey}
		\caption{Population distribution of mimicry rings, initialized with 2 prey species, 10k iterations}
		\label{fig:plot-2-prey}
	\end{figure}
}

%-----------------------------
%----- Unpalatable Species ------
%-----------------------------
\frame
{
	\frametitle{Only Unpalatable Species}
	\framesubtitle{Initial Configuration}

	\begin{table}[H]
	\centering
	\begin{tiny}
	\begin{spacing}{1.5}
	\begin{tabular}{|l|l|c|c|l|c|}
	  \hline
	   														&\multicolumn{3}{c|}{Prey configuration} 																	
	   														& \multicolumn{2}{c|}{Predator configuration} \\ \hline
	  \multirow{4}{*}{Population} & Rule110 (Unpalatable) & \parbox[c]{2.1em}{\includegraphics[scale=0.30]{../tex/images/CARule110}} 
	  																									& 150 & \multicolumn{2}{c|}{\multirow{4}{*}{\textbf{20 \(\downarrow\)}}} \\ \cline{2-4}
	  					 									& Rule30  (Unpalatable)& \parbox[c]{2.1em}{\includegraphics[scale=0.30]{../tex/images/CARule30}}  
	  					 																				& 150 & \multicolumn{2}{c|}{}\\ \cline{2-4}
	  					 									& Rule55  (Unpalatable)& \parbox[c]{2.1em}{\includegraphics[scale=0.30]{../tex/images/CARule55}}    
	  					 																				& 150 & \multicolumn{2}{c|}{}\\ \cline{2-4}
	  					 									& Rule190 (Unpalatable)& \parbox[c]{2.1em}{\includegraphics[scale=0.30]{../tex/images/CARule190}}& 150 & \multicolumn{2}{c|}{}\\ \hline
	  \multirow{2}{*}{Reproduction} & Age Limit & \multicolumn{2}{c|}{100}  & \multicolumn{2}{c|}{500} \\ \cline{2-6}
	  						 									& Interval  & \multicolumn{2}{c|}{1000} & \multicolumn{2}{c|}{2000} \\ \hline
	  \multirow{2}{*}{Mutation Rate} & Pattern   & \multicolumn{2}{c|}{0.05} & \multicolumn{2}{c|}{\multirow{2}{*}{0.3}} \\ \cline{2-4}
	  						 									 & Genome    & \multicolumn{2}{c|}{0.5}  & \multicolumn{2}{c|}{} \\ \hline
	  Demise Age	 									 & \multicolumn{3}{c|}{2000}							& \multicolumn{2}{c|}{\textbf{5000 \(\downarrow\)}} \\ \hline
	  Minimum Attack Age						 & \multicolumn{3}{c|}{} 						    & \multicolumn{2}{c|}{500} \\ \hline
	  \multirow{2}{*}{Memory Configuration} & \multicolumn{3}{c|}{} 					& Minimum & \textbf{4 \(\downarrow\)} \\ \cline{5-6}
	   																			& \multicolumn{3}{c|}{} 					& Maximum & 10 \\ \hline  
	\end{tabular}
	\end{spacing}
	\end{tiny}
	\caption{Agent configuration of 4 prey species all unpalatable.}
	\label{tab:config-table-4-prey-unpalatable}
	\end{table}	

}

\frame
{
	\frametitle{Only Unpalatable Species}
	\framesubtitle{Population vs. Time (10k)}

	\begin{figure}[H]
		\centering
		\includegraphics[scale=0.25]{../tex/images/simTime8k-4Prey-unp}
		\caption{Population distribution of mimicry rings(4 prey species all unpalatable)}
		\label{fig:plot-4-prey-unp}
	\end{figure}

}

\frame
{
	\frametitle{Only Unpalatable Species}
	\framesubtitle{Reduced Predator Memory\\ Population vs. Time (10k)}

	\begin{figure}[H]
		\centering
		\includegraphics[scale=0.25]{../tex/images/simTime6k-4Prey-unp-1-mem}
		\caption{Population distribution of mimicry rings. 4 prey, all unpalatable but reduced predator memory}
		\label{fig:plot-4-prey-unp-1-mem}
	\end{figure}
}

\subsection{Analysis}

\frame
{
	\frametitle{Analysis}
	\framesubtitle{Batesian Mimicry}

	\begin{itemize}
		\item Batesian Mimicry has taken effect, for all possible initial conditions.
			\begin{itemize}
				\item Every ring of unpalatable species there is a palatable ring.
			\end{itemize}
	\end{itemize}

	\begin{itemize}
		\item Start with palatable population, prey reaches extinction.\\
		\textbf{Reason:} No models to mimic for palatable species.
	\end{itemize}

	\begin{itemize}
		\item \textbf{Conclusion:} This model can simulate evolution of Batesian Mimicry.
	\end{itemize}

}

\frame
{
	\frametitle{Analysis}
	\framesubtitle{Mullerian Mimicry}

	\begin{quote}
		``Mullerian mimics converge into one large ring."
	\end{quote}
	
	\begin{itemize}
		\item Initialize simulation with 4 unpalatable species. No palatable ones.
		\item After 10k iteration all unpalatable species survive with dominance.
		\item \textbf{Reason:} Predator minimum memory configuration set to 4.
	\end{itemize}
	
	\textbf{New experiment:}
	\begin{itemize}
		\item Reduce predator memory to 1.\\
		\item \textbf{Observation:} Single large ring do not occur. 
		\item \textbf{Conclusion:} Similar to Franks and Noble.\\
		Multiple Mullerian mimics do not converge into one large ring.
	\end{itemize}	

}

\frame
{
	\frametitle{Analysis}
	\framesubtitle{Conclusion}

	\begin{itemize}
		\item Successful simulation of evolution of mimicry.
		\item Accurate simulation of mimicry ring.\\
		Diverse new rings and shift in their population.
		\item Proof of the theory of Turner: evolution of mimicry with punctuated equilibrium.
	\end{itemize}
}
\end{document}
