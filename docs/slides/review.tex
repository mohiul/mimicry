\section{Review}

\frame
{
	\begin{center}
		\LARGE A Review
	\end{center}
}

\frame
{
	\frametitle{Review}
}

\subsection{Evolutionary Computation}

\frame
{
	\frametitle{Evolutionary Computation}
	\framesubtitle{History}

	\begin{itemize}
		\item Turing (1948): ``Intelligent Machinery", expression ``genetical or evolutionary search"
		\item Bremermann (1962): Solve optimization problems. Minimize a real-valued function.
		\item Rechenberg (1964): Evolutionary strategies. Aerodynamic wing design.
		\item L. Fogel, Owens and Walsh (1965): Merged evolutionary computation and computational intelligence.
		\item Holland (1975): Genetic Algorithms, Schema theorem.
		\item Koza (1992): Genetic Programming. Automated method for creating computer program.
	\end{itemize}
}

\frame
{
	\frametitle{Evolutionary Algorithm}
	\framesubtitle{}

	\begin{itemize}
		\item Strategy: survival of the fittest.
		\item Optimize mathematical expressions or fitness function.
		\item Apply mutation and recombination.
		\item ``Variation operators creates the necessary diversity and there by facilitates novelty."
		\item ``Selection acts as a force pushing quality."
	\end{itemize}
}

\frame
{
	\frametitle{Genetic Algorithm}
	\framesubtitle{}

	\begin{itemize}
		\item 
	\end{itemize}
}

\frame
{
	\frametitle{Evolutionary Strategies}
	\framesubtitle{}

	\begin{itemize}
		\item Self adaptation of strategy parameters.
		\item Real-valued vectors
		\item Recombination process is discrete or intermediary.
		\item Mutation process uses Gaussian perturbation.
		\item Fast and good optimizer for real-valued optimization.
	\end{itemize}
}

\frame
{
	\frametitle{Evolutionary Programming}
	\framesubtitle{}

	\begin{itemize}
		\item Generate Artificial Intelligence by simulating evolution as a learning process.
		\item Intelligence: System should adapt its behavior to meet specified goal in the environment.
		\item Representation: real-valued vectors.
		\item Parent selection: Deterministic.
		\item Off springs are created via mutation.
		\item No recombination operator for evolutionary programming.
		\item Mutation is applied with Gaussian perturbation.
		\item Survivor selection is probabilistic.
	\end{itemize}
}

\frame
{
	\frametitle{Genetic Programming}
	\framesubtitle{}

	\begin{itemize}
		\item Applied to machine learning tasks such as prediction and classification.
		\item Attribute feature: Behaves similar to Neural Network.
		\item Algorithm itself is a very slow process.
		\item Non-linear chromosomes such as trees and graphs.
		\item Mutation is mostly avoided.
		\item Recombination operator work by exchanging trees.
		\item Parent selection: Fitness proportional.
		\item Survivor selection: Replacement of generations.
		\item Difference from genetic algorithm: the chromosomes are trees.
	\end{itemize}
}

\subsection{Artificial Life}

\frame
{
	\frametitle{Artificial Life}
	\framesubtitle{}

	\begin{itemize}
		\item Introduced by Christopher G. Langton.
		\item Organized at `International Conference on the Synthesis and Simulation of Living Systems', Los Alamos National laboratory 1987.
		\item Described by Taylor as a tool for biological inquiry.
		\item Wetware system:
			\begin{itemize}
				\item Similar to natural life. Indeed actually derived from natural life.
				\item Direct an artificial evolutionary process toward the production of RNA molecules.
			\end{itemize}
		\item Software system:
			\begin{itemize}
				\item Initial contribution from John von Neumann.
				\item Self reproducing, computation universal cellular automata.
			\end{itemize}
		\item Hardware system:
			\begin{itemize}
				\item Organism level.
				\item Can deal with organism's sensory and nervous system, its body and its environment.
				\item Geometric, mechanical, dynamical, thermal properties.
			\end{itemize}
	\end{itemize}
}

\frame
{
	\frametitle{Complex Adaptive System}
	\framesubtitle{}

	\begin{itemize}
		\item Special cases of complex systems.
		\item Diverse and made up of multiple interconnected elements.
		\item Adaptive as they have the capacity to change and learn from experience.
		\item Examples:
		\begin{itemize}
			\item Stock market, social insect and ant colonies.
			\item Biosphere and the ecosystem.
			\item Brain and the immune system.
			\item Cell and the developing embryo.
			\item Manufacturing businesses.
			\item Human social group-based endeavor in a cultural and social system.
			\item Large-scale online systems, collaborative tagging or social bookmarking systems.
		\end{itemize}
	\end{itemize}
}

\frame
{
	\frametitle{Complex Adaptive Systems}
	\framesubtitle{Criteria}

	\begin{itemize}
		\item ``All complex adaptive system involve large number of parts undergoing a kaleidoscopic
array of simultaneous non linear interactions."
		\item ``The impact of these systems in human affairs center on the aggregate behavior, the
behavior of the whole."
		\item ``The interactions evolve over time, as the parts adapt in an attempt to survive in the
environment provided by the other parts."
		\item ``Complex adaptive systems anticipate."
	\end{itemize}
}

\frame
{
	\frametitle{Echo}
	\framesubtitle{Criteria}

	\begin{itemize}
		\item Simplicity, thought experiment, primitive internal model.
		\item
		\item Fitness is an evolving criteria. It is provided by the site and other agents.
		\item ``The primitive mechanisms in Echo should have ready counter parts in all CAS".
		\item Flexible architecture to incorporate well studied mathematical models.
		\begin{itemize}
			\item Dawkins' biological arms race.
			\item Brower's survival of mimics.
			\item Wicksell's Triangle and overlapping generation models in economics.
			\item Prisoner's dilemma game in political science.
			\item Holland's two-armed bandits in operational research.
			\item Perelson's antigen-body matching in immunology.
		\end{itemize}
		\item Echo model should be amenable to mathematical analysis.
	\end{itemize}
}

\frame
{
	\frametitle{Ecology of Echo}
	\framesubtitle{Organization}

	\begin{itemize}
		\item 
	\end{itemize}
}

\frame
{
	\frametitle{Ecology of Echo}
	\framesubtitle{Resources and Sites}

	\begin{itemize}
		\item 
	\end{itemize}
}

\frame
{
	\frametitle{Ecology of Echo}
	\framesubtitle{Agents}

	\begin{itemize}
		\item Replication
		\begin{itemize}
			\item
		\end{itemize}
		\item Interactions
		\begin{itemize}
			\item
		\end{itemize}
		\item Combat
		\begin{itemize}
			\item
		\end{itemize}
		\item Trading
		\begin{itemize}
			\item
		\end{itemize}
		\item Mating
		\begin{itemize}
			\item
		\end{itemize}
	\end{itemize}
}

\frame
{
	\frametitle{Ecology of Echo}
	\framesubtitle{Results}

	\begin{itemize}
		\item Preston's Curve
		\begin{itemize}
			\item
		\end{itemize}
		\item Species-area scaling relation
		\begin{itemize}
			\item
		\end{itemize}
	\end{itemize}
}